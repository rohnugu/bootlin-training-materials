\section{Measuring}

\begin{frame}
\frametitle{Time measurement equipment: hardware}
\begin{itemize}
\item The best equipment is an oscilloscope, if you can afford one
\item Allows to time the "Power on" event (connected to a power rail),
      or any event (connected to a GPIO pin, for example), all this
      in a very accurate way.
\item Easy to write to a GPIO at all the stages of system booting
      (we will explain how to do it)
\item Some oscilloscopes are getting affordable. Example: Bitscope
      Pocket Analyzer (245 AUD, supported on Linux, \url{https://www.bitscope.com/product/BS10/})
\end{itemize}
\begin{center}
    % From https://openclipart.org/detail/28033/oscilloscope-by-mothinator
    \includegraphics[width=0.35\textwidth]{slides/boot-time-measuring/mothinator_Oscilloscope.pdf}
\end{center}
\end{frame}

\begin{frame}
\frametitle{Measuring with hardware: using an Arduino}
\begin{columns}
\column{0.85\textwidth}
  \url{https://arduino.cc}
  \begin{itemize}
  \item If you don't have an oscilloscope, an Arduino (or any general
        purpose MCU or MPU board) is a good solution too.
  \item The main strength of Arduino is its great ease of use and
        programming, plus all the hardware support libraries that are available.
  \item You can easily connect board pins to the Arduino analog pins, and
        watch their voltage.
  \item Arduino boards are Open Source Hardware. This project is
      definitely worth supporting!
  \end{itemize}
\column{0.15\textwidth}
  \begin{center}
  \tiny
  % From https://commons.wikimedia.org/wiki/File:Arduino_Logo.svg
  \includegraphics[width=0.5\textwidth]{slides/boot-time-measuring/arduino-logo.pdf}\\
  \vspace{1cm}
  Arduino Nano\\
  % From https://commons.wikimedia.org/wiki/File:Arduino_nano_isometr.jpg
  \includegraphics[width=\textwidth]{slides/boot-time-measuring/arduino-nano.jpg}\\
  Image credits: \url{https://commons.wikimedia.org/wiki/File:Arduino_nano_isometr.jpg}\\
  \vspace{1cm}
  \includegraphics[width=0.5\textwidth]{common/open-source-hardware-logo.pdf}
  \end{center}
\end{columns}
\end{frame}

\begin{frame}
\frametitle{Time measurement equipment: serial port}
\begin{columns}
\column{0.75\textwidth}
\small
\begin{itemize}
\item Useful when you don't have monitoring hardware, or don't want to make
      take any risk connecting wires to the hardware.
\item Usually relies on software which times messages received from the board's
      serial port (serial port absolutely required). Such software
      runs on a PC connected to the serial port.
\item On the board, requires a real serial port (directly connected to the CPU),
      immediately usable from the earliest parts of the boot process.
      Attaching a USB-to-serial dongle to a USB {\bf host} port on
      the device won't do: USB is available much later and messages
      go through more complex software stacks (loss of time accuracy).
\item Limitation: won't be able to time the "Power on" event in
      an accurate way. But acceptable as you can assume that
      the time to run the ROM code is constant.
\end{itemize}
\column{0.25\textwidth}
% From https://openclipart.org/detail/173570/serial-db9-female-by-deusinvictus-173570
\includegraphics[width=\textwidth]{slides/boot-time-measuring/serial_db9_female.pdf}\\
\vspace{1cm}
% From https://openclipart.org/detail/135721/usb-cable-by-gsagri04
\includegraphics[width=\textwidth]{slides/boot-time-measuring/GS_USB_Cable.pdf}
\end{columns}
\end{frame}

\begin{frame}
\frametitle{grabserial}
\url{https://elinux.org/Grabserial} (by Tim Bird)
\begin{itemize}
\item A Python script to add timestamps to messages received on a
      serial console.
\item Key advantage: starts counting very early (ROM code --- if not
      silent, bootstrap and bootloader)
\item Another advantage: no overhead on the target, because run on the host machine.
\item Drawbacks: may not be precise enough. Can't measure power up time.
\item Ubuntu package: \code{grabserial}\\
      Otherwise available on \url{https://github.com/tbird20d/grabserial/}
\end{itemize}
\end{frame}

\begin{frame}
\frametitle{Using grabserial}
\begin{center}
    \includegraphics[height=0.65\textheight]{slides/boot-time-measuring/using-grabserial.pdf}
\end{center}
{\small
{\bf Caution}: \code{grabserial} shows the arrival time of the
{\bf first character} of a line. This doesn't mean that the entire line
was received at that time.}
\end{frame}

\begin{frame}
\frametitle{grabserial tips}
\begin{itemize}
  \item You can interrupt \code{grabserial} manually (with
  \code{[Ctrl][c]}) when you have gone far enough.
  \item The \code{-m} ({\bf m}atch start pattern) and \code{-q} ({\bf
  q}uit pattern) options actually expect a regular expression.
  A normal string may not match in the middle of a line.
  \item Example: you may have to replace \code{-m "Starting kernel"} by
  \code{-m ".*Starting kernel.*"}.
  \item You can store a copy of the output to a file using the \code{-o}
        option. No need to copy / paste or redirect the output to keep it.
\end{itemize}
\end{frame}

\begin{frame}
\frametitle{Dedicated measuring resources}
Later, we will see specific resources for measuring time
\begin{itemize}
  \item \code{time} for measuring application time
  \item \code{strace} for application tracing
  \item \code{bootchartd} for measuring and tracing the execution of system services.
  \item More specifically, \code{systemd-analyze} if your system
	is started with {\em Systemd}.
  \item \kconfig{CONFIG_PRINTK_TIME} and \code{initcall_debug} for
        tracing and timestamping kernel code and functions.
\end{itemize}
\end{frame}

\setuplabframe {Measuring time}
{
Measuring time with software
\begin{itemize}
\item Setting up \code{grabserial}
\item Modify the video player to log a notification
      after the first frame is processed.
\item Time the various components of boot time through messages
      written to the serial console.
\end{itemize}
}
