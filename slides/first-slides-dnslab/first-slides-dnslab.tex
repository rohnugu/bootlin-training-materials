\titleframe

\begin{frame}{Heejun Roh}
  \begin{itemize}
  \item Deparment Head and Assistant Professor of {\bf Korea University Sejong Campus}.
  \item Education
  \begin{itemize}
    \item Ph.D. in Computer Science and Engineering, Graduate School, Korea University
    \item M.S. in Computer Science and Engineering, Graduate School, Korea University
    \item B.S. in Computer Science and Engineering and Mathematics (Double Major), Korea University
  \end{itemize}
  \item Research Professor of {\em Future Network Center, Korea University} (2017-2019)
  \item Research Interests: Network Traffic Monitoring and Analysis, Security and Privacy in Wireless and Mobile Networks, RF-powered Computing and Networking, etc.
  \item Office: Room 207, Sci. Tech. Bldg II, KU Sejong
  \item E-mail: \code{hjroh@korea.ac.kr} or +82-10-9852-6156
  \item Websites:
  \begin{itemize}
    \item Laboratory: \url{https://dnslab.korea.ac.kr/}
    \item Personal: \url{https://netlab.korea.ac.kr/hjroh}
  \end{itemize}
  \end{itemize}
\end{frame}


\begin{frame}{Syllabus}
  \begin{itemize}
  \item Class Description
    \begin{itemize}
      \item In this course, we study essentials of system programming, which is essential for cyber security major. By understanding the details of Linux operating system focusing on the Linux kernel, system calls, and system software development, we study fundamental concepts, principles, and techniques toward secure software design and implementation.
    \end{itemize}
  \item Correlation with core competences and departmental objectives
    \begin{itemize}
      \item Based on understanding system programming essentials, we try to correlate the ideas with problem solving and professionalism in pragmatic practice. That is, research capability in computing and pragmatic practices are highly related with this course.
    \end{itemize} 
  \item Study Objectives
    \begin{itemize}
      \item From this course, students will study the Linux system architecture, how to build, setup, and use a toolchain, how to interact with hardware devices, and how to develop system software in Linux.
    \end{itemize}
  \item Course Objectives
    \begin{itemize}
      \item We focus on the embedded Linux system development, rather than the advanced system programming topics such as advanced file I/O, process management, threading, file and directory management, signals, time, etc. Of course, this approach has pros and cons, but your understanding on embedded Linux system will be deep.
    \end{itemize}  
  \end{itemize}
\end{frame}

\begin{frame}{Syllabus}
  \begin{itemize}
  \item Prerequisite Subjects
    \begin{itemize}
      \item Intermediate level of Linux experience is required. If you forgot how to use Linux, please check `The Missing Semester of Your CS Education' \url{https://missing.csail.mit.edu/} where you can find the Korean version. Or if you prefer Korean language, 생활코딩's Linux Playlist would be useful: \url{https://www.youtube.com/watch?v=DsG-JWrFJTc}.

      \item Programming experiences in the C programming language are required. Also, taking our major's Computer Architecture course (AICS221) and Operating System course (AICS302) are highly recommended.
    \end{itemize}
  \item Textbook \& References: We will primarily use slide sets, but the following books can be helpful.
    \begin{enumerate}
      \item 윤현주, Linux: 기초에서 시스템 프로그래밍까지, 2nd Ed., 도서출판 홍릉, 2021. (for beginners only)
      \item Robert Love, Linux System Programming, 2nd Ed., O'Reilly, 2013. (한국어판有 - 김영근譯, 리눅스 시스템 프로그래밍, 2nd Ed., 한빛미디어, 2017.)
      \item Rodolfo Giometti, GNU/Linux Rapid Embedded Programming, 1st Ed., Packt Publishing, 2017. (한국어판有 - 정병혁譯, GNU/Linux 쾌속 임베디드 프로그래밍, 1st Ed., 에이콘출판사, 2018.)
      \item 김동현, 디버깅을 통해 배우는 리눅스 커널의 구조와 원리: 라즈베리 파이로 따라하면서 쉽게 이해할 수 있는 리눅스 커널 Vol. 1 and 2, 1st Ed., 위키북스, 2020.
      \item 김동현, 시스템 소프트웨어 개발을 위한 Arm 아키텍처의 구조와 원리: Armv8-A와 Armv7-A로 배우는 시스템 반도체와 전기자동차 시스템 개발의 핵심, 1st Ed., 위키북스, 2023.
      \item Brian Ward, How Linux Works: What Every Superuser Should Know, 3rd Ed., No Starch Press, 2021. (ISBN: 9781718500402; 한국어판有 - 유하영, 전우영譯, 슈퍼 사용자라면 반드시 알아야 할 리눅스 작동법, 2nd Ed., BJ퍼블릭, 2015; 한국어판 ISBN: 9791186697054)
      \item Randal E. Bryant and David R. O'Hallaron, Computer Systems: A Programmer's Perspective - Global Edition, 3rd Ed., Pearson, 2015. (ISBN: 9780134092669; 한국어판有 - 김형신譯, 컴퓨터 시스템, 퍼스트북刊; 한국어판 ISBN: 9791185475219)
    \end{enumerate}
  \item Homework
    \begin{itemize}
      \item Many programming projects should be done. (Once for 1-2 weeks)
      \item No late submission for every homework/projects.      
    \end{itemize}
  \end{itemize}
\end{frame}


%% \ifdefempty{\trainer}의 바로 뒤에 {}는 \trainer가 정의되지 않았을 때 동작을 말하는 것이고, 그 뒤의 {}는 \trainer가 정의되었을 때의 동작임. 따라서 아래 코드는 \trainer가 정의되었을 때 (trainer 이름).tex를 읽어와 삽입하는 기능임
\ifdefempty{\trainer}{}{
  \input{../slides/first-slides-dnslab/\trainer.tex}
}

%% If the materials a generated for a real session, not for the website

\ifdefempty{\trainer}{}{
  \begin{frame}
  \frametitle{Electronic copies of these documents}
     \begin{itemize}
        \item Electronic copies of your particular version of the
              materials are available on:\\
              {\scriptsize \url{\sessionurl}} \\
        \item You can download and open these documents to follow
	      lectures and labs, to look for explanations given earlier
              by the trainer and to copy and paste text during labs.
        \item This specific URL will remain available for a long time.
	      This way, you can always access the exact instructions
              corresponding to the labs performed in this session.
        \item If you are interested in the latest versions of our
	      training materials, visit the description of each
              course on \url{https://bootlin.com/training/}.
    \end{itemize}
  \end{frame}
}