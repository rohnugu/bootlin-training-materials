\section{Distro Layers}

\subsection{Distro Layers}

\begin{frame}{Distro layers}
  \begin{center}
    \includegraphics[height=0.8\textheight]{slides/yocto-layer-distro/yocto-layer-distro.pdf}
  \end{center}
\end{frame}

\begin{frame}
  \frametitle{Distro layers}
  \begin{itemize}
    \item You can create a new distribution by using a Distro layer.
    \item This corresponds to the settings that have an impact on your
          packages. You can also decide to use Musl or Glibc, Wayland
          or X11, systemd or sysvinit\dots
    \item A distribution layer allows to change the defaults that are provided by
      \code{openembedded-core} or \code{poky}.
    \item It is useful to distribute changes that have been made in
      \code{local.conf}
    \item Note: Poky is a rather bloated distribution, mainly meant to
          be used for testing. It's not necessarily a good starting
          point to optimize the root filesystem for your own platform.
  \end{itemize}
\end{frame}

\begin{frame}
  \frametitle{Best practice}
  \begin{itemize}
    \item A distro layer is used to provide policy configurations for
      a custom distribution.
    \item It is a best practice to separate the distro layer from the
      custom layers you may create and use.
    \item It often contains:
      \begin{itemize}
        \item Configuration files.
        \item Specific classes (for example to sign images)
        \item Distribution specific recipes: initialization scripts,
          splash screen\dots
      \end{itemize}
  \end{itemize}
\end{frame}

\begin{frame}[fragile]
  \frametitle{Creating a Distro layer}
  \begin{itemize}
    \item The configuration file for the distro layer is
      \code{conf/distro/<distro>.conf}
    \item This file must define the \yoctovar{DISTRO} variable.
    \item It is possible to inherit configuration from an existing
      distro layer.
    \item You can also use all the \code{DISTRO_*} variables.
    \item Use \code{DISTRO = "<distro>"} in \code{local.conf} to use
      your distro configuration.
  \end{itemize}
  \begin{block}{}
    \begin{minted}[fontsize=\small]{sh}
require conf/distro/poky.conf

DISTRO = "distro"
DISTRO_NAME = "distro description"
DISTRO_VERSION = "1.0"

MAINTAINER = "..."
    \end{minted}
  \end{block}
\end{frame}

\begin{frame}
  \frametitle{\code{DISTRO_FEATURES}}
  \begin{itemize}
    \item Lists the features the distribution will enable (SSL, WiFi,
          Bluetooth\dots).
    \item As for \yoctovar{MACHINE_FEATURES}, this is used by package
      recipes to enable or disable functionalities.
    \item For example, the \code{bluetooth} feature:
      \begin{itemize}
        \item Asks the \code{bluez} daemon to be built and added to
          the image.
        \item Enables bluetooth support in \code{ConnMan}.
      \end{itemize}
    \item \yoctovar{COMBINED_FEATURES} provides the list of features that
      are enabled in both \yoctovar{MACHINE_FEATURES} and \yoctovar{DISTRO_FEATURES}.
  \end{itemize}
\end{frame}

\begin{frame}
  \frametitle{Toolchain selection}
  \begin{itemize}
    \item The toolchain selection is controlled by the \yoctovar{TCMODE}
      variable.
    \item It defaults to \code{"default"}.
    \item The \code{conf/distro/include/tcmode-${TCMODE}.inc} file is
      included.
    \begin{itemize}
      \item This configures the toolchain to use by defining preferred
        providers and versions for recipes such as \code{gcc},
        \code{binutils}, \code{*libc}\dots
    \end{itemize}
    \item The providers' recipes define how to compile or/and install
      the toolchain.
    \item Toolchains can be built by the build system or external
      (rarely used because toolchains are fast to rebuild thanks
       to the shared state cache).
  \end{itemize}
\end{frame}

\begin{frame}
  \frametitle{Sample files}
  \begin{itemize}
    \item A distro layer often contains \code{sample files}, used as
      templates to build key configurations files.
    \item Example of \code{sample files}:
      \begin{itemize}
        \item \code{bblayers.conf.sample}
        \item \code{local.conf.sample}
      \end{itemize}
    \item In \code{Poky}, they are in \code{meta-poky/conf/}.
    \item The \yoctovar{TEMPLATECONF} variable controls where to find the
      samples.
    \item It is set in \code{${OEROOT}/.templateconf}.
      \begin{itemize}
        \item \yoctovar{OEROOT} is the directory that contains the
          \code{oe-init-build-env} script.
      \end{itemize}
  \end{itemize}
\end{frame}
