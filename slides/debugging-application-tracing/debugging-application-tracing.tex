\section{Application Tracing}

\begin{frame}[fragile]{strace}
  \begin{columns}[T]
  \column{0.75\textwidth}
  \small
  System call tracer - \url{https://strace.io}
  \begin{itemize}
  \item Available on all GNU/Linux systems\\
        Can be built by your cross-compiling toolchain generator or by your build system.
  \item Allows to see what any of your processes is doing: accessing files, allocating memory...
        Often sufficient to find simple bugs.
  \item Usage:\\
    \code{strace <command>} (starting a new process)\\
    \code{strace -f <command>} ({\bf f}ollow child processes too)\\
    \code{strace -p <pid>} (tracing an existing process)\\
    \code{strace -c <command>} (time statistics per system call)
    \code{strace -e <expr> <command>} (use {\bf e}xpression for advanced filtering)
  \end{itemize}
  See \href{https://man7.org/linux/man-pages/man1/strace.1.html}{the strace manual} for details.
  \column{0.25\textwidth}
  \includegraphics[height=0.7\textheight]{common/strace-mascot.png}\\
  \tiny Image credits: \url{https://strace.io/}
  \end{columns}
\end{frame}

\begin{frame}[fragile]{strace example output}
  \includegraphics[height=0.75\textheight]{common/strace-output.pdf}\\
  Hint: follow the open file descriptors returned by \code{open()}.
  This tells you what files are handled by further system calls.
\end{frame}

\begin{frame}[fragile]
  \frametitle{strace -c example output}
  \includegraphics[height=0.8\textheight]{common/strace-c-output.pdf}
\end{frame}

\begin{frame}{ltrace}
  A tool to trace {\bf shared} library calls used by a program and all the signals
  it receives
  \begin{itemize}
  \item Very useful complement to \code{strace}, which shows only system
    calls.
  \item Of course, works even if you don't have the sources
  \item Allows to filter library calls with regular expressions, or
    just by a list of function names.
  \item With the \code{-S} option it shows system calls too!
  \item Also offers a summary with its \code{-c} option.
  \item Manual page: \url{https://linux.die.net/man/1/ltrace}
  \item Works better with {\em glibc}. \code{ltrace} used to be broken
        with {\em uClibc} (now fixed), and is not supported
        with {\em Musl} (Buildroot 2022.11 status).
  \end{itemize}
  See \url{https://en.wikipedia.org/wiki/Ltrace} for details
\end{frame}

\begin{frame}[fragile]{ltrace example output}
  \scriptsize
  \begin{block}{}
\begin{verbatim}
# ltrace  ffmpeg -f video4linux2 -video_size 544x288 -input_format mjpeg -i /dev
/video0 -pix_fmt rgb565le -f fbdev /dev/fb0
__libc_start_main([ "ffmpeg", "-f", "video4linux2", "-video_size"... ] <unfinished ...>
setvbuf(0xb6a0ec80, nil, 2, 0)                   = 0
av_log_set_flags(1, 0, 1, 0)                     = 1
strchr("f", ':')                                 = nil
strlen("f")                                      = 1
strncmp("f", "L", 1)                             = 26
strncmp("f", "h", 1)                             = -2
strncmp("f", "?", 1)                             = 39
strncmp("f", "help", 1)                          = -2
strncmp("f", "-help", 1)                         = 57
strncmp("f", "version", 1)                       = -16
strncmp("f", "buildconf", 1)                     = 4
strncmp("f", "formats", 1)                       = 0
strlen("formats")                                = 7
strncmp("f", "muxers", 1)                        = -7
strncmp("f", "demuxers", 1)                      = 2
strncmp("f", "devices", 1)                       = 2
strncmp("f", "codecs", 1)                        = 3
...
\end{verbatim}
\end{block}
\end{frame}

\begin{frame}[fragile]{ltrace summary}
  Example summary at the end of the ltrace output (\code{-c} option)
  \scriptsize
  \begin{block}{}
\begin{verbatim}

% time     seconds  usecs/call     calls      function
------ ----------- ----------- --------- --------------------
 52.64    5.958660     5958660         1 __libc_start_main
 20.64    2.336331     2336331         1 avformat_find_stream_info
 14.87    1.682895         421      3995 strncmp
  7.17    0.811210      811210         1 avformat_open_input
  0.75    0.085290         584       146 av_freep
  0.49    0.055150         434       127 strlen
  0.29    0.033008         660        50 av_log
  0.22    0.025090         464        54 strcmp
  0.20    0.022836       22836         1 avformat_close_input
  0.16    0.017788         635        28 av_dict_free
  0.15    0.016819         646        26 av_dict_get
  0.15    0.016753         440        38 strchr
  0.13    0.014536         581        25 memset
...
------ ----------- ----------- --------- --------------------
100.00   11.318773                  4762 total
\end{verbatim}
  \end{block}
\end{frame}


\begin{frame}
  \frametitle{Hooking Library Calls}
  \begin{itemize}
    \item In order to do some more complex library call hooks, one can use
          the {\em LD\_PRELOAD} environment variable.
    \item {\em LD\_PRELOAD} is used to specify a shared library that will be
          loaded before any other library by the dynamic loader.
    \item Allows to intercept all library calls by preloading another library.
    \begin{itemize}
      \item Overrides libraries symbols that have the same name.
      \item Allows to redefine only a few specific symbols.
      \item "Real" symbol can still be loaded and used with \code{dlsym} (\manpage{dlsym}{3})
    \end{itemize}
    \item Used by some debugging/tracing libraries ({\em libsegfault},
          {\em libefence})
    \item Works for C and C++.
  \end{itemize}
\end{frame}

\begin{frame}[fragile]
  \frametitle{{\em LD\_PRELOAD} example}
  \begin{itemize}
    \item Library snippet that we want to preload using {\em LD\_PRELOAD}:
  \end{itemize}
  \begin{block}{}
    \begin{minted}[fontsize=\small]{c}
#include <string.h>
#include <unistd.h>

ssize_t read(int fd, void *data, size_t size) {
  memset(data, 0x42, size);
  return size;
}
    \end{minted}
  \end{block}
  \begin{itemize}
    \item Compilation of the library for {\em LD\_PRELOAD} usage:
  \end{itemize}
  \begin{block}{}
    \begin{minted}[fontsize=\small]{console}
$ gcc -shared -fPIC -o my_lib.so my_lib.c
    \end{minted}
  \end{block}

  \begin{itemize}
    \item Preloading the new library using {\em LD\_PRELOAD}:
  \end{itemize}
  \begin{block}{}
    \begin{minted}[fontsize=\small]{console}
$ LD_PRELOAD=./my_lib.so ./exe
    \end{minted}
  \end{block}
\end{frame}

\begin{frame}[fragile]
  \frametitle{uprobes}
  \begin{itemize}
    \item {\em uprobe} is a mechanism offered by the kernel allowing to trace
          userspace code.
    \item Tracepoints can be added dynamically on any userspace symbol
    \begin{itemize}
      \item Internally patches the \code{.text} section with breakpoints
        that are handled by the kernel trace system
    \end{itemize}
    \item Exposed by file \code{/sys/kernel/debug/tracing/uprobe_events}
    \item Often wrapped up by other tools (\code{perf}, \code{bcc} for
          instance).
    \item \kdochtml{trace/uprobetracer}
  \end{itemize}
\end{frame}

\begin{frame}[fragile]
  \frametitle{The {\em perf} tool}
  \begin{itemize}
    \item {\em perf} tool was started as a tool to profile application under
          Linux using performance counters (\manpage{perf}{1}).
    \item It became much more than that and now allows to manage tracepoints,
          kprobes and uprobes.
    \item {\em perf} can profile both user-space and kernel-space execution.
    \item {\em perf} is based on the \code{perf_event} interface that is
          exposed by the kernel.
    \item Provides a set of operations, each having specific arguments (see
          {\em perf} help).
    \begin{itemize}
      \item \code{stat}, \code{record}, \code{report}, \code{top}, \code{annotate}, \code{ftrace}, \code{list}, \code{probe}, etc
    \end{itemize}
  \end{itemize}
\end{frame}

\begin{frame}[fragile]
  \frametitle{Using {\em perf record}}
  \begin{itemize}
    \item {\em perf record} allows to record performance events per-thread,
          per-process and per-cpu basis.
    \item Kernel needs to be configured with \kconfigval{CONFIG_PERF_EVENTS}{y}.
    \item This is the first command that needs to be run to gather data from
          program execution and output them into \code{perf.data}.
    \item \code{perf.data} file can then be analyzed using \code{perf annotate}
          and \code{perf report}.
    \begin{itemize}
      \item Useful on embedded systems to analyze data on another computer.
    \end{itemize}
  \end{itemize}
\end{frame}

\begin{frame}[fragile]
  \frametitle{Probing userspace functions}
  \begin{itemize}
    \item List lines number that can be probed in a specific
          executable/function:
  \end{itemize}
  \begin{block}{}
    \begin{minted}[fontsize=\scriptsize]{C}
$ perf probe --source=<source_dir> -x my_app -L my_func
    \end{minted}
  \end{block}
  \begin{itemize}
    \item Create tracepoints on user-space library/executable functions:
  \end{itemize}
  \begin{block}{}
    \begin{minted}[fontsize=\scriptsize]{C}
$ perf probe -x /lib/libc.so.6 printf
$ perf probe -x app my_func:3 my_var
$ perf probe -x app my_func%return ret=%r0
    \end{minted}
  \end{block}
  \begin{itemize}
  \item Record the execution of these tracepoints:
  \end{itemize}
  \begin{block}{}
    \begin{minted}[fontsize=\scriptsize]{C}
$ perf record -e probe_app:my_func -e probe_libc:printf
    \end{minted}
  \end{block}
\end{frame}

\setuplabframe
{Application tracing}
{
  Analyzing of application interactions
  \begin{itemize}
    \item Analyze dynamic library calls from an application using
            {\em ltrace}.
    \item Using {\em strace} to analyze program syscalls.
  \end{itemize}
}
