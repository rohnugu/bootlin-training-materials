\section{Embedded Linux application development}

\begin{frame}{Contents}
  \begin{itemize}
  \item Application development
    \begin{itemize}
    \item Developing applications on embedded Linux
    \item Building your applications
    \end{itemize}
  \item Debugging and analysis tools
    \begin{itemize}
    \item Debuggers
    \item Remote debugging
    \item Tracing and profiling
    \end{itemize}
  \end{itemize}
\end{frame}

\subsection{Developing applications on embedded Linux}

\begin{frame}{Application development}
  \begin{itemize}
  \item An embedded Linux system is just a normal Linux system, with
    usually a smaller selection of components
  \item In terms of application development, developing on embedded
    Linux is exactly the same as developing on a desktop Linux system
  \item All existing skills can be re-used, without any particular
    adaptation
  \item All existing libraries, either third-party or in-house, can be
    integrated into the embedded Linux system
    \begin{itemize}
    \item Taking into account, of course, the limitation of the
      embedded systems in terms of performance, storage and memory
    \end{itemize}
  \item Application development could start on x86, even before
      the hardware is available.
  \end{itemize}
\end{frame}

\begin{frame}{Leverage existing libraries and languages}
  \begin{itemize}
  \item Many developers getting started with embedded Linux limit
    themselves to C, sometimes C++, and the C/C++ standard library.
  \item However, there are a lot of libraries and languages that can
    help you accelerate and simplify your application development
    \begin{itemize}
    \item Compiled languages like Rust and Go are increasingly popular
    \item Interpreted languages, especially Python
    \item Higher-level libraries: Qt, Glib, Boost, and many more
    \end{itemize}
  \item Make sure to evaluate what is the right choice for your
    project, but pay attention to
    \begin{itemize}
    \item Footprint and performance on low-end platforms
    \item Use well-maintained and well-known technologies
    \end{itemize}
  \end{itemize}
\end{frame}

\begin{frame}{Building your applications/libraries}
  \begin{itemize}
  \item Even for simple applications or libraries, make use of a build
    system
    \begin{itemize}
    \item \href{https://cmake.org/}{CMake}
    \item \href{https://mesonbuild.com/}{Meson}
    \end{itemize}
  \item This will simplify
    \begin{itemize}
    \item the build process of your application
    \item the life of developers joining your project
    \item the packaging of your application into an embedded Linux
      build system
    \end{itemize}
  \end{itemize}
\end{frame}

\begin{frame}[fragile]{Getting started with {\em meson}}
  \begin{block}{Minimal {\tt meson.build}}
\begin{verbatim}
project('example', 'c')
executable('demo', 'main.c')
\end{verbatim}
  \end{block}

  \begin{block}{{\tt meson.build} for multiple programs and source files}
\begin{verbatim}
project('example', 'c')
src_demo1 = ['demo1.c', 'foo1.c']
executable('demo1', src_demo1)
src_demo2 = ['demo2.c', 'foo2.c']
executable('demo2', src_demo2)
\end{verbatim}
  \end{block}
\end{frame}

\begin{frame}[fragile]{Options with {\em meson}}
  \begin{block}{{\tt meson\_options.txt}}
\begin{verbatim}
option('demo-debug', type : 'feature', value : 'disabled')
\end{verbatim}
  \end{block}

  \begin{block}{{\tt meson.build}}
\begin{verbatim}
project('tutorial', 'c')
demo_c_args = []
if get_option('demo-debug').enabled()
   demo_c_args += '-DDEBUG'
endif
executable('demo', 'main.c', c_args: demo_c_args)
\end{verbatim}
  \end{block}
\end{frame}

\begin{frame}[fragile]{Library dependencies with {\em meson}}
  \begin{block}{{\tt meson.build}}
\begin{verbatim}
project('tutorial', 'c')
gtkdep = dependency('gtk+-3.0')
executable('demo', 'main.c', dependencies : gtkdep)
\end{verbatim}
  \end{block}
  The dependency \code{gtk+-3.0} is searched using \code{pkg-config}.
\end{frame}

\subsection{Debugging}

\begin{frame}
  \frametitle{GDB: GNU Project Debugger}
  \fontsize{11}{11}\selectfont
  \begin{columns}[T]
    \column{0.8\textwidth}
    \begin{itemize}
    \item The debugger on GNU/Linux, available for most embedded
      architectures.
    \item Supported languages: C, C++, Pascal, Objective-C, Fortran,
      Ada...
    \item Command-line interface
    \item Integration in many graphical IDEs
    \item Can be used to
      \begin{itemize}
      \item control the execution of a running program, set
        breakpoints or change internal variables
      \item to see what a program was doing when it crashed: post
        mortem analysis
      \end{itemize}
    \item \url{https://www.gnu.org/software/gdb/}
    \item \url{https://en.wikipedia.org/wiki/Gdb}
    \item New alternative: {\em lldb} (\url{https://lldb.llvm.org/})\\
      from the LLVM project.
    \end{itemize}
    \column{0.2\textwidth}
    \includegraphics[width=0.9\textwidth]{common/gdb.png}
  \end{columns}
\end{frame}

\begin{frame}[fragile]
  \frametitle{GDB crash course (1/3)}
  \begin{itemize}
    \item GDB is used mainly to debug a process by starting it with {\em gdb}
    \begin{itemize}
      \item \code{$ gdb <program> <args>}
    \end{itemize}
    \item GDB can also be attached to running processes using the program PID
    \begin{itemize}
      \item \code{$ gdb -p <pid>}
    \end{itemize}
    \item When using GDB to start a program, the program needs to be run with
    \begin{itemize}
      \item \code{(gdb) run}
    \end{itemize}
  \end{itemize}
\end{frame}

\begin{frame}
  \frametitle{GDB crash course (2/3)}
  \small
  A few useful GDB commands
  \begin{itemize}
  \item \code{break foobar} (\code{b})\\
    Put a breakpoint at the entry of function \code{foobar()}
  \item \code{break foobar.c:42}\\
    Put a breakpoint in \code{foobar.c}, line 42
  \item \code{print var}, \code{print $reg} or \code{print task->files[0].fd} (\code{p})\\
    Print the variable \code{var}, the register \code{$reg} or a more
    complicated reference. GDB can also nicely display structures with all
    their members
  \item \code{info registers}\\
    Display architecture registers
  \end{itemize}
\end{frame}

\begin{frame}
  \frametitle{GDB crash course (3/3)}
  \small
  \begin{itemize}
  \item \code{continue} (\code{c})\\
    Continue the execution after a breakpoint
  \item \code{next} (\code{n})\\
    Continue to the next line, stepping over function calls
  \item \code{step} (\code{s})\\
    Continue to the next line, entering into subfunctions
  \item \code{stepi} (\code{si})\\
    Continue to the next instruction
  \item \code{finish}\\
    Execute up to function return
  \item \code{backtrace} (\code{bt})\\
    Display the program stack
  \end{itemize}
\end{frame}

\ifthenelse{\equal{\training}{debugging}}
{
\begin{frame}
  \frametitle{GDB advanced commands (1/3)}
  \small
  \begin{itemize}
    \item \code{info threads} (\code{i threads})\\
      Display the list of threads that are available
    \item \code{info breakpoints} (\code{i b})\\
      Display the list of breakpoints/watchpoints
    \item \code{delete <n>} (\code{d <n>})\\
      Delete breakpoint <n>
    \item \code{thread <n>} (\code{t <n>})\\
      Select thread number <n>
    \item \code{frame <n>} (\code{f <n>})\\
      Select a specific frame from the backtrace, the number being the one
      displayed when using \code{backtrace} at the beginning of each line
  \end{itemize}
\end{frame}

\begin{frame}
  \frametitle{GDB advanced commands (2/3)}
  \small
  \begin{itemize}
    \item \code{watch <variable>} or \code{watch \*<address>}\\
      Add a watchpoint on a specific variable/address.
    \item \code{print variable = value} (\code{p variable = value})\\
      Modify the content of the specified variable with a new value
    \item \code{break if condition == value}\\
      Break only if the specified condition is true
    \item \code{watch if condition == value}\\
      Trigger the watchpoint only if the specified condition is true
    \item \code{x/<n><u> <address>}\\
      Display memory at the provided address. \code{n} is the amount of memory to
      display, \code{u} is the type of data to be displayed (\code{b/h/w/g}).
      Instructions can be displayed using the \code{i} type.
  \end{itemize}
\end{frame}

\begin{frame}
  \frametitle{GDB advanced commands (3/3)}
  \small
  \begin{itemize}
    \item \code{list <expr>}\\
      Display the source code associated to the current program counter location.
    \item \code{disassemble <location,start_offset,end_offset>} (\code{disas})\\
      Display the assembly code that is currently executed.
    \item \code{p function(arguments)}\\
      Execute a function using GDB. NOTE: be careful of any side effects that
      may happen when executing the function
    \item \code{p $newvar = value}\\
      Declare a new gdb variable that can be used locally or in command sequence
    \item \code{define <command_name>}\\
      Define a new command sequence. GDB will prompt for the sequence of
      commands.
  \end{itemize}
\end{frame}
}
{
\subsection{Remote debugging}
}

\begin{frame}
  \frametitle{Remote debugging}
  \begin{itemize}
  \item In a non-embedded environment, debugging takes place using \code{gdb}
    or one of its front-ends.
  \item \code{gdb} has direct access to the binary and libraries compiled
    with debugging symbols.
  \item However, in an embedded context, the target platform
    environment is often too limited to allow direct debugging with
    \code{gdb} (2.4 MB on x86).
  \item Remote debugging is preferred
    \begin{itemize}
    \item \code{ARCH-linux-gdb} is used on the development workstation, offering
      all its features.
    \item \code{gdbserver} is used on the target system (only 400 KB
      on arm).
    \end{itemize}
  \end{itemize}
  \begin{center}
    \includegraphics[width=0.5\textwidth]{common/gdb-vs-gdbserver.pdf}
  \end{center}
\end{frame}

\begin{frame}
  \frametitle{Remote debugging: architecture}
  \begin{center}
    \includegraphics[width=\textwidth]{common/gdb-vs-gdbserver-architecture.pdf}
  \end{center}
\end{frame}

\begin{frame}
  \frametitle{Remote debugging: usage}
  \begin{itemize}
  \item On the target, run a program through \code{gdbserver}.\\
    Program execution will not start immediately.\\
    \code{gdbserver :<port> <executable> <args>}
    \code{gdbserver /dev/ttyS0 <executable> <args>}
  \item Otherwise, attach \code{gdbserver} to an already running program:\\
    \code{gdbserver --attach :<port> <pid>}
  \item Then, on the host, start \code{ARCH-linux-gdb <executable>},\\
    and use the following \code{gdb} commands:
    \begin{itemize}
    \item To tell \code{gdb} where shared libraries are:\\
      \code{gdb> set sysroot <library-path>} (typically path to build space without \code{lib/})
    \item To connect to the target:\\
      \code{gdb> target remote <ip-addr>:<port>} (networking)\\
      \code{gdb> target remote /dev/ttyUSB0} (serial link)
    \end{itemize}
  \end{itemize}
\end{frame}

\begin{frame}
  \frametitle{Coredumps for post mortem analysis}
  \begin{itemize}
  \item When an application crashes due to a {\em segmentation fault}
    and the application was not under control of a debugger, we get no
    information about the crash
  \item Fortunately, Linux can generate a \code{core} file that
    contains the image of the application memory at the moment of the
    crash, and gdb can use this \code{core} file to let us analyze the
    state of the crashed application
  \item On the target
    \begin{itemize}
    \item Use \code{ulimit -c unlimited} in the shell starting the
      application, to enable the generation of a \code{core} file
      when a crash occurs
    \item The output name for the coredump file can be modified using
      \code{/proc/sys/kernel/core_pattern}.
    \item See \manpage{core}{5}
    \end{itemize}
  \item On the host
    \begin{itemize}
    \item After the crash, transfer the \code{core} file from the target to
      the host, and run
      \code{ARCH-linux-gdb -c core-file application-binary}
    \end{itemize}
  \end{itemize}
\end{frame}


\subsection{Tracing and profiling}

\begin{frame}[fragile]{strace}
  \begin{columns}[T]
  \column{0.75\textwidth}
  \small
  System call tracer - \url{https://strace.io}
  \begin{itemize}
  \item Available on all GNU/Linux systems\\
        Can be built by your cross-compiling toolchain generator or by your build system.
  \item Allows to see what any of your processes is doing: accessing files, allocating memory...
        Often sufficient to find simple bugs.
  \item Usage:\\
    \code{strace <command>} (starting a new process)\\
    \code{strace -f <command>} ({\bf f}ollow child processes too)\\
    \code{strace -p <pid>} (tracing an existing process)\\
    \code{strace -c <command>} (time statistics per system call)
    \code{strace -e <expr> <command>} (use {\bf e}xpression for advanced filtering)
  \end{itemize}
  See \href{https://man7.org/linux/man-pages/man1/strace.1.html}{the strace manual} for details.
  \column{0.25\textwidth}
  \includegraphics[height=0.7\textheight]{common/strace-mascot.png}\\
  \tiny Image credits: \url{https://strace.io/}
  \end{columns}
\end{frame}

\begin{frame}[fragile]{strace example output}
  \includegraphics[height=0.75\textheight]{common/strace-output.pdf}\\
  Hint: follow the open file descriptors returned by \code{open()}.
  This tells you what files are handled by further system calls.
\end{frame}

\begin{frame}[fragile]
  \frametitle{strace -c example output}
  \includegraphics[height=0.8\textheight]{common/strace-c-output.pdf}
\end{frame}

\begin{frame}{ltrace}
  A tool to trace {\bf shared} library calls used by a program and all the signals
  it receives
  \begin{itemize}
  \item Very useful complement to \code{strace}, which shows only system
    calls.
  \item Of course, works even if you don't have the sources
  \item Allows to filter library calls with regular expressions, or
    just by a list of function names.
  \item With the \code{-S} option it shows system calls too!
  \item Also offers a summary with its \code{-c} option.
  \item Manual page: \url{https://linux.die.net/man/1/ltrace}
  \item Works better with {\em glibc}. \code{ltrace} used to be broken
        with {\em uClibc} (now fixed), and is not supported
        with {\em Musl} (Buildroot 2022.11 status).
  \end{itemize}
  See \url{https://en.wikipedia.org/wiki/Ltrace} for details
\end{frame}

\begin{frame}[fragile]{ltrace example output}
  \scriptsize
  \begin{block}{}
\begin{verbatim}
# ltrace  ffmpeg -f video4linux2 -video_size 544x288 -input_format mjpeg -i /dev
/video0 -pix_fmt rgb565le -f fbdev /dev/fb0
__libc_start_main([ "ffmpeg", "-f", "video4linux2", "-video_size"... ] <unfinished ...>
setvbuf(0xb6a0ec80, nil, 2, 0)                   = 0
av_log_set_flags(1, 0, 1, 0)                     = 1
strchr("f", ':')                                 = nil
strlen("f")                                      = 1
strncmp("f", "L", 1)                             = 26
strncmp("f", "h", 1)                             = -2
strncmp("f", "?", 1)                             = 39
strncmp("f", "help", 1)                          = -2
strncmp("f", "-help", 1)                         = 57
strncmp("f", "version", 1)                       = -16
strncmp("f", "buildconf", 1)                     = 4
strncmp("f", "formats", 1)                       = 0
strlen("formats")                                = 7
strncmp("f", "muxers", 1)                        = -7
strncmp("f", "demuxers", 1)                      = 2
strncmp("f", "devices", 1)                       = 2
strncmp("f", "codecs", 1)                        = 3
...
\end{verbatim}
\end{block}
\end{frame}

\begin{frame}[fragile]{ltrace summary}
  Example summary at the end of the ltrace output (\code{-c} option)
  \scriptsize
  \begin{block}{}
\begin{verbatim}

% time     seconds  usecs/call     calls      function
------ ----------- ----------- --------- --------------------
 52.64    5.958660     5958660         1 __libc_start_main
 20.64    2.336331     2336331         1 avformat_find_stream_info
 14.87    1.682895         421      3995 strncmp
  7.17    0.811210      811210         1 avformat_open_input
  0.75    0.085290         584       146 av_freep
  0.49    0.055150         434       127 strlen
  0.29    0.033008         660        50 av_log
  0.22    0.025090         464        54 strcmp
  0.20    0.022836       22836         1 avformat_close_input
  0.16    0.017788         635        28 av_dict_free
  0.15    0.016819         646        26 av_dict_get
  0.15    0.016753         440        38 strchr
  0.13    0.014536         581        25 memset
...
------ ----------- ----------- --------- --------------------
100.00   11.318773                  4762 total
\end{verbatim}
  \end{block}
\end{frame}


\begin{frame}{ftrace}
  \begin{itemize}
  \item In-kernel {\em tracing} functionality
  \item Can trace
    \begin{itemize}
    \item Well-defined trace locations in the kernel, called {\em
        tracepoints}, identifying important events in the kernel:
      scheduling, interrupts, etc.
    \item Arbitrary functions in the kernel
    \item Arbitrary functions in user-space applications
    \end{itemize}
  \item Low-overhead and optimized tracing
  \item Accessible using the dedicated {\em tracefs} filesystem
  \item \code{trace-cmd} is a higher-level CLI tool to use {\em
      ftrace}
  \item Can be used to understand overall system activity (what is my
    system doing?) as well as narrow down specific performance
    issues
  \item \url{https://www.kernel.org/doc/Documentation/trace/ftrace.txt}
  \item \url{https://www.trace-cmd.org/}
  \end{itemize}
\end{frame}

\begin{frame}{kernelshark}
  \begin{itemize}
  \item Visualization tool for {\em ftrace} traces
  \item \url{https://kernelshark.org/}
  \end{itemize}
  \begin{center}
    \includegraphics[height=0.6\textheight]{slides/sysdev-application-development/kernelshark.png}
  \end{center}
\end{frame}

\begin{frame}{perf}
  \begin{itemize}
  \item {\em instrument CPU performance counters, tracepoints, kprobes, and uprobes}
  \item Directly included in the Linux kernel source code: \kfile{tools/perf}
  \item Began as a tool for using the performance counters in Linux,
    and has had various enhancements to add tracing capabilities
  \item Supports a list of measurable events: hardware events (cycle
    count, L1 cache hits/miss, page faults), software events
    (tracepoints)
  \item \url{https://perf.wiki.kernel.org}
  \end{itemize}
\end{frame}

\begin{frame}{perf examples}
  \begin{itemize}
  \item List all currently known events\\
    \code{perf list}
  \item List scheduler tracepoints\\
    \code{perf list 'sched:*'}
  \item CPU counter statistics for the specified command\\
    \code{perf stat <command>}
  \item CPU counter statistics for the entire system, for 5 seconds\\
    \code{perf stat -a sleep 5}
  \item Profiling: sample on-CPU functions for the specified command, at 99 Hertz\\
    \code{perf record -F 99 <command>}
  \item Tracing: trace all context-switches via sched tracepoint, until Ctrl-C\\
    \code{perf record -e sched:sched_switch -a}
  \item Many more at \url{https://www.brendangregg.com/perf.html}
  \end{itemize}
\end{frame}

\begin{frame}{perf GUI: hotspot}
  \begin{columns}[T]
    \column{0.4\textwidth}
    \begin{itemize}
    \item Hotspot - the Linux perf GUI for performance analysis
    \item The main feature of hotspot is visualizing a \code{perf.data} file graphically
    \item \href{https://github.com/KDAB/hotspot}{github.com/KDAB/hotspot}
    \end{itemize}
    \column{0.6\textwidth}
    \includegraphics[width=\textwidth]{slides/sysdev-application-development/hotspot.png}
  \end{columns}
\end{frame}

\begin{frame}{gprof}
  \begin{itemize}
  \item Application-level profiler
  \item Part of {\em binutils}
  \item Requires passing gcc \code{-pg} option at build/link time
  \item Run your program normally, it automatically generates a
    \code{gmon.out} file when exiting
  \item Use the \code{gprof} tool on \code{gmon.out} to extract
    profiling data
  \item \url{http://sourceware.org/binutils/docs/gprof/}
  \end{itemize}
\end{frame}

\begin{frame}[fragile]{gprof example}
  \begin{columns}
    \column{0.6\textwidth}
    \begin{block}{}
      {\tiny
\begin{verbatim}
$ ./test-gprof
$ gprof test-gprof gmon.out
Flat profile:

Each sample counts as 0.01 seconds.
  %   cumulative   self              self     total
 time   seconds   seconds    calls   s/call   s/call  name
 35.31      7.46     7.46        1     7.46    13.92  func1
 34.03     14.65     7.19        1     7.19     7.19  func2
 30.57     21.11     6.46        1     6.46     6.46  new_func1
  0.09     21.13     0.02                             main
[...]
\end{verbatim}
      }
    \end{block}
    \column{0.4\textwidth}
    \begin{center}
      \includegraphics[height=0.6\textheight]{slides/sysdev-application-development/gprof2dot.pdf}\\
      {\small Generated with \href{https://github.com/jrfonseca/gprof2dot}{gprof2dot}}
    \end{center}
  \end{columns}
\end{frame}

\subsection{Memory debugging}

\input{../common/valgrind.tex}

\begin{frame}{Debugging resources}
  \begin{columns}[T]
  \column{0.65\textwidth}
  \begin{itemize}
  \item Brendan Gregg
    \href{https://www.brendangregg.com/systems-performance-2nd-edition-book.html}{Systems
      performance} book
  \item Brendan Gregg
    \href{https://www.brendangregg.com/linuxperf.html}{Linux
      Performance} page
  \item Bootlin's "Linux debugging, profiling, tracing and performance
        analysis" training course and free training materials
        (250 pages): \url{https://bootlin.com/training/debugging/}.
  \end{itemize}
  \column{0.35\textwidth}
  \includegraphics[height=0.6\textheight]{slides/debugging-principles/cloud_word.png}
  \end{columns}
\end{frame}

\setuplabframe
{Application development and debugging}
{
  \begin{itemize}
  \item Creating an application that uses an I2C-connected joystick to
    control an audio player.
  \item Setting up an IDE to develop and remotely debug an
    application.
  \item Using {\em strace}, {\em ltrace}, {\em gdbserver} and {\em
      perf} to debug/investigate buggy applications on the embedded
    board.
  \end{itemize}
}
