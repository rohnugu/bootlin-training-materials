\subsection{Definition and Components}

\begin{frame}
  \frametitle{Toolchain definition (1)}
  \begin{itemize}
  \item The usual development tools available on a GNU/Linux
    workstation is a {\bf native toolchain}
  \item This toolchain runs on your workstation and generates code for
    your workstation, usually x86
  \item For embedded system development, it is usually impossible or not
    interesting to use a native toolchain
    \begin{itemize}
    \item The target is too restricted in terms of storage and/or memory
    \item The target is very slow compared to your workstation
    \item You may not want to install all development tools on your target.
    \end{itemize}
  \item Therefore, {\bf cross-compiling toolchains} are generally
    used. They run on your workstation but generate code for your
    target.
  \end{itemize}
\end{frame}

\begin{frame}
  \frametitle{Toolchain definition (2)}
  \begin{center}
    \includegraphics[width=0.8\textwidth]{slides/sysdev-toolchains-definition/cross-toolchain.pdf}
  \end{center}
\end{frame}

\begin{frame}{Architecture tuple and toolchain prefix}
  \begin{itemize}
  \item Many UNIX/Linux build mechanisms rely on {\em architecture
      tuple} names to identify machines.
  \item Examples: \code{arm-linux-gnueabihf},
    \code{mips64el-linux-gnu}, \code{arm-vendor-none-eabihf}
  \item These tuples are 3 or 4 parts:
    \begin{enumerate}
    \item The architecture name: \code{arm}, \code{riscv},
      \code{mips64el}, etc.
    \item Optionally, a vendor name, which is a free-form string
    \item An operating system name, or \code{none} when not targeting
      an operating system
    \item The ABI/C library (see later)
    \end{enumerate}
  \item This tuple is used to:
    \begin{itemize}
    \item configure/build software for a given platform
    \item as a prefix of cross-compilation tools, to differentiate
      them from the native toolchain
      \begin{itemize}
      \item \code{gcc} $\rightarrow$ native compiler
      \item \code{arm-linux-gnueabihf-gcc} $\rightarrow$ cross-compiler
      \end{itemize}
    \end{itemize}
  \end{itemize}
\end{frame}

\begin{frame}
  \frametitle{Components of gcc toolchains}
  \begin{center}
    \includegraphics[width=0.8\textwidth]{slides/sysdev-toolchains-definition/components.pdf}
  \end{center}
\end{frame}

\begin{frame}
  \frametitle{Binutils}
  \begin{itemize}
  \item {\bf Binutils} is a set of tools to generate and manipulate
    binaries (usually with the ELF format) for a given CPU architecture
    \begin{itemize}
    \item \code{as}, the assembler, that generates binary code from
      assembler source code
    \item \code{ld}, the linker
    \item \code{ar}, \code{ranlib}, to generate \code{.a} archives
     (static libraries)
    \item \code{objdump}, \code{readelf}, \code{size}, \code{nm},
      \code{strings}, to inspect binaries. Very useful analysis tools!
    \item \code{objcopy}, to modify binaries
    \item \code{strip}, to strip parts of binaries that are just needed
      for debugging (reducing their size).
    \end{itemize}
  \item GNU Binutils: \url{https://www.gnu.org/software/binutils/}, GPL license
  \end{itemize}
\end{frame}

\begin{frame}
  \frametitle{C/C++ compiler}
  \begin{columns}[T]
    \column{0.8\textwidth}
    \begin{itemize}
    \item GCC: GNU Compiler Collection, the famous free software compiler
    \item \url{https://gcc.gnu.org/}
    \item Can compile C, C++, Ada, Fortran, Java, Objective-C,
      Objective-C++, Go, etc. Can generate code for a large number of CPU
      architectures, including x86, ARM, RISC-V, and many others.
    \item Available under the GPL license, libraries under the GPL with
      linking exception.
    \end{itemize}
    \column{0.2\textwidth}
    \includegraphics[width=0.7\textwidth]{slides/sysdev-toolchains-definition/gcc.png}
  \end{columns}
\end{frame}

\begin{frame}
  \frametitle{Kernel headers (1)}
  \begin{columns}[T]
    \column{0.7\textwidth}
    \begin{itemize}
    \item The C standard library and compiled programs need to interact with the kernel
      \begin{itemize}
      \item Available system calls and their numbers
      \item Constant definitions
      \item Data structures, etc.
      \end{itemize}
    \item Therefore, compiling the C standard library requires kernel headers, and many
      applications also require them.
    \item Available in \code{<linux/...>} and \code{<asm/...>} and a few
      other directories corresponding to the ones visible in
      \kdir{include/uapi} and in \code{arch/<arch>/include/uapi} in the kernel sources
    \item The kernel headers are extracted from the kernel sources using
      the \code{headers_install} kernel Makefile target.
    \end{itemize}
    \column{0.3\textwidth}
    \includegraphics[width=\textwidth]{slides/sysdev-toolchains-definition/kernel-headers.pdf}
  \end{columns}
\end{frame}

\begin{frame}[fragile]
  \frametitle{Kernel headers (2)}
  \begin{itemize}
  \item System call numbers, in \code{<asm/unistd.h>}
\begin{verbatim}
#define __NR_exit         1
#define __NR_fork         2
#define __NR_read         3
\end{verbatim}
  \item Constant definitions, here in \code{<asm-generic/fcntl.h>},
    included from \code{<asm/fcntl.h>}, included from
    \code{<linux/fcntl.h>}
\begin{verbatim}
#define O_RDWR 00000002
\end{verbatim}
\item Data structures, here in \code{<asm/stat.h>} (used by the
\code{stat} command)
\begin{verbatim}
struct stat {
    unsigned long st_dev;
    unsigned long st_ino;
    [...]
};
\end{verbatim}
\end{itemize}
\end{frame}

\begin{frame}[fragile]
  \frametitle{Kernel headers (3)}
  The kernel to user space interface is {\bf backward compatible}
  \begin{itemize}
  \item Kernel developers are doing their best to {\bf never}
        break existing programs when the kernel is upgraded.
        Otherwise, users would stick to older kernels, which
        would be bad for everyone.
  \item Hence, binaries generated with a toolchain using kernel headers
        older than the running kernel will work without problem, but
        won't be able to use the new system calls, data structures, etc.
  \item Binaries generated with a toolchain using kernel headers
        newer than the running kernel might work only if they don't use
        the recent features, otherwise they will break.
  \end{itemize}
  What to remember: updating your kernel shouldn't break your programs;
  it's usually fine to keep an old toolchain as long is it works fine
  for your project.
\end{frame}

\begin{frame}
  \frametitle{C/C++ compiler}
  \begin{columns}[T]
    \column{0.8\textwidth}
    \begin{itemize}
    \item GCC: GNU Compiler Collection, the famous free software compiler
    \item \url{https://gcc.gnu.org/}
    \item Can compile C, C++, Ada, Fortran, Java, Objective-C,
      Objective-C++, Go, etc. Can generate code for a large number of CPU
      architectures, including x86, ARM, RISC-V, and many others.
    \item Available under the GPL license, libraries under the GPL with
      linking exception.
    \end{itemize}
    \column{0.2\textwidth}
    \includegraphics[width=0.7\textwidth]{slides/sysdev-toolchains-definition/gcc.png}
  \end{columns}
\end{frame}

\begin{frame}
  \frametitle{C standard library}
  \begin{columns}[T]
    \column{0.6\textwidth}
    \begin{itemize}
    \item The C standard library is an essential component of a Linux
          system.
      \begin{itemize}
      \item Interface between the applications and the kernel
      \item Provides the well-known standard C API to ease application
        development
      \end{itemize}
    \item Several C standard libraries are available:
      {\em glibc}, {\em uClibc}, {\em musl}, {\em klibc}, {\em
        newlib}...
    \item The choice of the C standard library must be made at
      cross-compiling toolchain generation time, as the GCC compiler is
      compiled against a specific C standard library.
    \end{itemize}
    \column{0.4\textwidth}
    \includegraphics[width=\textwidth]{slides/sysdev-toolchains-definition/Linux_kernel_System_Call_Interface_and_uClibc.pdf}\\
    {\tiny Source: Wikipedia (\url{https://bit.ly/2zrGve2})}
  \end{columns}
\end{frame}

\begin{frame}
  \frametitle{glibc}
  \begin{columns}[T]
    \column{0.7\textwidth}
    \begin{itemize}
    \item License: LGPL
    \item C standard library from the GNU project
    \item Designed for performance, standards compliance and portability
    \item Found on all GNU / Linux host systems
    \item Of course, actively maintained
    \item By default, quite big for small embedded systems.
      On armv7hf, version 2.31: \code{libc}: 1.5 MB, \code{libm}: 432
      KB, source: \url{https://toolchains.bootlin.com}
    \item \url{https://www.gnu.org/software/libc/}
    \end{itemize}
    \vfill
    \column{0.3\textwidth}
    \minipage[c][0.8\textheight][s]{\columnwidth}
    \includegraphics[width=\textwidth]{slides/sysdev-toolchains-definition/heckert_gnu_white.pdf}
    \vfill
    \tiny \href{https://en.wikipedia.org/wiki/File:Heckert_GNU_white.svg}{Image source}
    \endminipage
  \end{columns}
\end{frame}

\begin{frame}
  \frametitle{uClibc-ng}
  \begin{itemize}
  \item \url{https://uclibc-ng.org/}
  \item A continuation of the old uClibc project, license: LGPL
  \item Lightweight C standard library for small embedded systems
    \begin{itemize}
    \item High configurability: many features can be enabled or
      disabled through a menuconfig interface.
    \item Supports most embedded architectures, including MMU-less
          ones (ARM Cortex-M, Blackfin, etc.). The only standard library
          supporting ARM noMMU.
    \item No guaranteed binary compatibility. May need to
      recompile applications when the library configuration changes.
    \item Some features may be implemented later than on glibc (real-time,
          floating-point operations...)
    \item Focus on size (RAM and storage) rather than performance
    \item Size on armv7hf, version 1.0.34:
      \code{libc}: 712 KB, source: \url{https://toolchains.bootlin.com}
    \end{itemize}
    \item Actively supported, but Yocto Project stopped supporting it.
  \end{itemize}
\end{frame}

\begin{frame}
  \frametitle{musl C standard library}
  \begin{columns}[T]
    \column{0.85\textwidth}
      \url{https://www.musl-libc.org/}
      \begin{itemize}
      \item A lightweight, fast and simple standard library for embedded systems
      \item Created while uClibc's development was stalled
      \item In particular, great at making small static executables,
	    which can run anywhere, even on a system built
            from another C standard library.
      \item More permissive license (MIT), making it easier to release
            static executables. We will talk about the requirements
            of the LGPL license (glibc, uClibc) later.
      \item Supported by build systems such as Buildroot and Yocto
        Project.
      \item Used by the Alpine Linux lightweight distribution
        (\url{https://www.alpinelinux.org/})
      \item Size on armv7hf, version 1.2.0:
        \code{libc}: 748 KB, source: \url{https://toolchains.bootlin.com}
      \end{itemize}
    \column{0.15\textwidth}
    \includegraphics[width=\textwidth]{slides/sysdev-toolchains-definition/musl.png}
  \end{columns}
\end{frame}

\begin{frame}
  \frametitle{Other smaller C libraries}
  \begin{itemize}
  \item Several other smaller C libraries exist, but they do not
    implement the full POSIX interface required by most Linux
    applications
  \item They can run only relatively simple programs, typically to
    make very small static executables and run in very small root
    filesystems.
  \item Therefore not commonly used in most embedded Linux systems
  \item Choices:
    \begin{itemize}
    \item Newlib, \url{https://sourceware.org/newlib/}, maintained by
      Red Hat, used mostly in Cygwin, in bare metal and in small POSIX
      RTOS.
    \item Klibc, \url{https://en.wikipedia.org/wiki/Klibc}, from the
      kernel community, designed to implement small executables for
      use in an {\em initramfs} at boot time.
    \end{itemize}
  \end{itemize}
\end{frame}

\begin{frame}
  \frametitle{Advice for choosing the C standard library}
  \begin{itemize}
  \item Advice to start developing and debugging your applications
    with {\em glibc}, which is the most standard solution
  \item If you have size constraints, try to compile your app and then
    the entire filesystem with {\em uClibc} or {\em musl}
    \begin{itemize}
    \item The size advantage of {\em uClibc} or {\em musl}, which used
      to be a significant argument, is less relevant with today's
      storage capacities.
    \item Smaller binaries and filesystems remain useful when optimizing
      boot time, though, typically booting on a filesystem loaded in
      RAM, and to reduce the size of container and virtual machine images
      (one of the use cases of Alpine Linux).
    \end{itemize}
  \item If you run into trouble, it could be because of missing
    features in the C standard library.
  \item In case you wish to make static executables, {\em musl} will
    be an easier choice in terms of licensing constraints.
  \end{itemize}
\end{frame}

\begin{frame}{Linux vs. bare-metal toolchain}
  \begin{itemize}
  \item A {\bf Linux toolchain}
    \begin{itemize}
    \item is a toolchain that includes a Linux-ready C standard library, which
      uses the Linux system calls to implement system services
    \item can be used to build Linux user-space applications, but also
      bare-metal code (firmware, bootloader, Linux kernel)
    \item is identified by the \code{linux} OS identifier in the
      toolchain tuple: \code{arm-linux},
      \code{arm-none-linux-gnueabihf}
    \end{itemize}
  \item A {\bf bare metal toolchain}
    \begin{itemize}
    \item is a toolchain that does not include a C standard library, or a very
      minimal one that isn't tied to a particular operating
      system
    \item can be used to build only bare-metal code (firmware,
      bootloader, Linux kernel)
    \item is identified by the \code{none} OS identifier in the
      toolchain tuple: \code{arm-none-eabi}, \code{arm-none-none-eabi}
      (vendor is \code{none}, OS is \code{none})
    \end{itemize}
  \end{itemize}
\end{frame}

\begin{frame}{An alternate compiler suite: LLVM}
  \begin{itemize}
  \item Most Embedded Linux projects use toolchains based on GNU
    project: GCC compiler, binutils, GDB debugger
  \item The LLVM project has been developing an alternative compiler
    suite:
    \begin{itemize}
    \item Clang, C/C++ compiler, \url{https://clang.llvm.org/}
    \item LLDB, debugger, \url{https://lldb.llvm.org/}
    \item LLD, linker, \url{https://lld.llvm.org/}
    \item and more, see \url{https://llvm.org/}
    \end{itemize}
  \item While they are used by several high-profile projects, they are
    not yet in widespread use in most Embedded Linux projects.
  \item Initially had better code optimization and diagnostics than
    GCC, but thanks to having competition, GCC has improved
    significantly in this area.
  \item Available under MIT/BSD licenses
  \item \url{https://en.wikipedia.org/wiki/LLVM}
  \end{itemize}
\end{frame}
