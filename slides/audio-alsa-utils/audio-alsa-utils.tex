\subsection{alsa-utils}

\begin{frame}{alsa-utils}
  \begin{itemize}
  \item \code{alsa-utils} is a repository of tools to interact with
    ALSA devices
  \item \url{https://github.com/alsa-project/alsa-utils}
  \end{itemize}
\end{frame}

\begin{frame}{Controls}
  \begin{itemize}
  \item \code{alsamixer} provides a \code{ncurse} based graphical
    interface to modify sound cards controls.
  \item \code{amixer} is a command line tool to set controls.
    \begin{itemize}
    \item The \code{scontrols} and \code{controls} commands list the
      available controls.
    \item The \code{contents} commands list the
      available controls and shows their content.
    \item The \code{cset} and \code{sset} commands allows modifying
      the controls.
    \item The \code{cget} and \code{sget} commands show the content of
      a specific control.
    \end{itemize}
  \item \code{alsactl} is a tool that can save the control values to a
    file and restore them from a file.
  \end{itemize}
\end{frame}

\begin{frame}{Playback and capture}
  \begin{itemize}
  \item \code{speaker-test} can generate tones or noises to play on
    specific channels with a specified rate.
  \item \code{aplay} plays an audio file. It is able to set many
    stream parameters.
  \item \code{arecord} can record an audio stream to a file.
  \end{itemize}
\end{frame}

\setupdemoframe
{Userspace tools}
{
  Using userspace tools to:
  \begin{itemize}
  \item configure sound card controls
  \item load and store default values for controls
  \item play sound
  \item record
  \end{itemize}
}
