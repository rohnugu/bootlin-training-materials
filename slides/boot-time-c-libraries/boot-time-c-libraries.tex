\begin{frame}
  \frametitle{glibc}
  \begin{columns}
    \column{0.7\textwidth}
    \begin{itemize}
    \item License: LGPL
    \item C library from the GNU project
    \item Designed for performance, standards compliance and portability
    \item Found on all GNU / Linux host systems
    \item Of course, actively maintained
    \item By default, quite big for small embedded systems.
      On armv7hf, version 2.31: \code{libc}: 1.5 MB, \code{libm}: 432
      KB, source: \url{https://toolchains.bootlin.com}
    \item \url{https://www.gnu.org/software/libc/}
    \end{itemize}
    \vfill
    \column{0.3\textwidth}
    \minipage[c][0.8\textheight][s]{\columnwidth}
    \includegraphics[width=\textwidth]{slides/boot-time-c-libraries/heckert_gnu_white.pdf}
    \vfill
    \tiny Image: \url{https://bit.ly/2EzHl6m}
    \endminipage
  \end{columns}
\end{frame}

\begin{frame}
  \frametitle{uClibc-ng}
  \begin{itemize}
  \item \url{https://uclibc-ng.org/}
  \item A continuation of the old uClibc project, license: LGPL
  \item Lightweight C library for small embedded systems
    \begin{itemize}
    \item High configurability: many features can be enabled or
      disabled through a menuconfig interface.
    \item Supports most embedded architectures, including MMU-less
          ones (ARM Cortex-M, Blackfin, etc.). The only library
          supporting ARM noMMU.
    \item No guaranteed binary compatibility. May need to
      recompile applications when the library configuration changes.
    \item Some features may be implemented later than on glibc (real-time,
          floating-point operations...)
    \item Focus on size (RAM and storage) rather than performance
    \item Size on armv7hf, version 1.0.34:
      \code{libc}: 712 KB, source: \url{https://toolchains.bootlin.com}
    \end{itemize}
    \item Actively supported, but Yocto Project stopped supporting it.
  \end{itemize}
\end{frame}

\begin{frame}
  \frametitle{musl C library}
  \begin{columns}
    \column{0.85\textwidth}
      \url{https://www.musl-libc.org/}
      \begin{itemize}
      \item A lightweight, fast and simple library for embedded systems
      \item Created while uClibc's development was stalled
      \item In particular, great at making small static executables,
	    which can run anywhere, even on a system built
            from another C library.
      \item More permissive license (MIT), making it easier to release
            static executables. We will talk about the requirements
            of the LGPL license (glibc, uClibc) later.
      \item Supported by build systems such as Buildroot and Yocto
        Project.
      \item Used by the Alpine Linux lightweight distribution
        (\url{https://www.alpinelinux.org/})
      \item Size on armv7hf, version 1.2.0:
        \code{libc}: 748 KB, source: \url{https://toolchains.bootlin.com}
      \end{itemize}
    \column{0.15\textwidth}
    \includegraphics[width=\textwidth]{slides/boot-time-c-libraries/musl.png}
  \end{columns}
\end{frame}

\begin{frame}{glibc vs uclibc-ng vs musl - small static executables}
  Let's compile and strip a \code{hello.c} program {\bf statically} and
compare the size
  \begin{itemize}
    \item With musl 1.2.0:\\
          {\bf 9,084} bytes
    \item With uclibc-ng 1.0.34:\\
          {\bf 21,916} bytes.
    \item With glibc 2.31:\\
          {\bf 431,140} bytes
  \end{itemize}
  Tests run with \code{gcc 10.0.2} toolchains for \code{armv7-eabihf}\\
  (from \url{https://toolchains.bootlin.com})
\end{frame}

\begin{frame}{glibc vs uclibc vs musl - more realistic example}
  Let's compile and strip BusyBox 1.32.1 {\bf statically}\\
  (with the \code{defconfig} configuration) and compare the size
  \begin{itemize}
    \item With musl 1.2.0:\\
          {\bf 1,176,744} bytes
    \item With uclibc-ng 1.0.34:\\
          {\bf 1,251,080} bytes.\\
    \item With glibc 2.31:\\
          {\bf 1,852,912} bytes
  \end{itemize}
  Notes:
  \begin{itemize}
    \item Tests run with \code{gcc 10.0.2} toolchains for \code{armv7-eabihf}
    \item BusyBox is automatically compiled with \code{-Os} and stripped.
    \item Compiling with shared libraries will mostly eliminate size differences
  \end{itemize}
\end{frame}

\begin{frame}
  \frametitle{Other smaller C libraries}
  \begin{itemize}
  \item Several other smaller C libraries have been developed, but
    none of them have the goal of allowing the compilation of large
    existing applications
  \item They can run only relatively simple programs,
	typically to make very small static executables and run
	in very small root filesystems.
  \item Choices:
    \begin{itemize}
    \item Newlib, \url{https://sourceware.org/newlib/},
	  maintained by Red Hat, used mostly in Cygwin, in bare metal
          and in small POSIX RTOS.
    \item Klibc, \url{https://en.wikipedia.org/wiki/Klibc},
          from the kernel community, designed to implement small executables
          for use in an {\em initramfs} at boot time.
    \end{itemize}
  \end{itemize}
\end{frame}

\begin{frame}
  \frametitle{Advise for choosing the C library}
  \begin{itemize}
  \item Advice to start developing and debugging your applications with
        {\em glibc}, which is the most standard solution, and is best
        supported by debugging tools ({\em ltrace} not supported by {\em
        musl} in Buildroot, for example).
  \item Then, when everything works, if you have size constraints, try to compile
        your app and then the entire filesystem with {\em uClibc} or {\em musl}.
  \item If you run into trouble, it could be because of missing features
        in the C library.
  \item In case you wish to make static executables, {\em musl} will be
        an easier choice in terms of licensing constraints.
	The binaries will be smaller too.
        Note that static executables built with a given C library
        can be used in a system with a different C library.
  \end{itemize}
\end{frame}
