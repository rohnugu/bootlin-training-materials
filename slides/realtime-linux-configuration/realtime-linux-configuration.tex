\section{Configuring the system}

\begin{frame}
	\frametitle{Configuration}
	\begin{itemize}
		\item The Linux Kernel has a lot of available configurations
		\item Some are dedicated to performance
			\begin{itemize}
				\item Focus on improving the most-likely scenario
				\item Have the best average performance
				\item Optimize throughput and low-latency
				\item Usually have a slow-path, which is non-deterministic
			\end{itemize}
		\item Some are dedicated to security and hardening
		\item Some can be useful for Deterministic Behaviour
	\end{itemize}
\end{frame}

\begin{frame}
	\frametitle{CPU Pinning}
	\begin{itemize}
		\item The Linux Kernel Scheduler allows setting constraints about the CPU cores that are allowed
			to run each task
		\item This can be useful for lots of purposes:
			\begin{itemize}
				\item Make sure that a process won't be migrated to another core
				\item Dedicate cores for specific tasks
				\item Optimize the data-path if a process deals with data handled by a specific CPU core
				\item Ease the job of the scheduler's CPU load-balancer, whose complexity grows non-linearly with the number of CPUs
			\end{itemize}
		\item This mechanism is called the \code{cpu affinity} of a process
		\item The \code{cpuset} subsystem and the \code{sched_setaffinity} syscall are used to select the CPUs
		\item Use \code{taskset -p <mask> <cmd>} to start a new process on the given CPUs
	\end{itemize}
\end{frame}

% isolcpus
\begin{frame}
	\frametitle{CPU Isolation}
	\begin{itemize}
		\item Users can pin processes to CPU cores through the cpu affinity mechanism
		\item But the kernel might also schedule other processes on these CPUs
		\item \code{isolcpus} can be passed on the kernel commandline
		\item Isolated CPUs will not be used by the scheduler
		\item The only way to run processes on these CPUs is with cpu affinity
		\item Very useful when RT processes coexist with non-RT processes
	\end{itemize}
	\code{isolcpus=0,2,3}
\end{frame}

\begin{frame}
	\frametitle{CPU Isolation - cpusets}
	\begin{itemize}
		\item \code{cpuset} is a mechanism allowing to subdivide the CPU scheduling pool
		\item They are created at runtime, through the \code{cpusetfs}
			\begin{itemize}
				\item \code{mount -t cpuset none /dev/cpuset}
			\end{itemize}
		\item cpusets are created at will in the cpuset main directory
			\begin{itemize}
				\item \code{mkdir /dev/cpuset/rt-set}
				\item \code{mkdir /dev/cpuset/non-rt-set}
			\end{itemize}
		\item Each cpuset is assigned a pool of cpu cores
			\begin{itemize}
				\item \code{/bin/echo 2,3 > /dev/cpuset/rt-set}
				\item \code{/bin/echo 0,1 > /dev/cpuset/non-rt-set}
			\end{itemize}
		\item We can then select which task gets to run in each cpuset
			\begin{itemize}
				\item \code{while read i; do /bin/echo $i; done < /dev/cpuset/tasks > /dev/cpuset/nontrt-set/tasks}
				\item \code{/bin/echo $$ > /dev/cpuset/rt-set/tasks}
			\end{itemize}
		\item You can run tasks in a given set with \code{cgexec -g cpuset:rt-set ...}
	\end{itemize}
\end{frame}

% irq assignment
\begin{frame}
	\frametitle{IRQ affinity}
	\begin{itemize}
		\item Interrupts are handled by a specific CPU core
		\item The default CPU that handles interrupts is the CPU 0
		\item On Multi-CPU systems, it can be good to balance interrupt handling between CPUs
		\item Similarly, we might also want to prevent CPUs from handling external interrupts
		\item IRQs can be pinned to CPUs by tweaking \code{/proc/irq/XX/smp_affinity}
		\item The \code{irqbalance} tool monitors and distributes the irq affinty to spread the load across CPUs
		\item Use the \code{IRQBALANCE_BANNED_CPUS} environment variable to make \code{irqbalance} ignore some CPUs
		\item The \code{irqaffinity} cmdline parameter can also be used
	\end{itemize}
\end{frame}

\begin{frame}
	\frametitle{RCU Callbacks and Workqueues}
	\begin{center}\textbf{RCU}\end{center}
	\begin{itemize}
		\item \textbf{R}ead \textbf{C}opy \textbf{U}pdate
		\item Synchronisation mechanism that can deferred object reclamation
		\item Deffered reclamation can be executed on any CPU \code{RCU callbacks}
		\item We can preven CPU cores from running RCU callbacks with \code{rcu_nocbs=<cpus> rcu_nocb_poll}
	\end{itemize}
	\begin{center}\textbf{Workqueues}\end{center}
	\begin{itemize}
		\item Deferred execution mechanism
		\item Can be pinned to CPUs in \code{/sys/devices/virtual/workqueue/cpumask}
	\end{itemize}
\end{frame}
%TODO RCU/workqueues

% scheduling classes
\begin{frame}
	\frametitle{The Linux Kernel Scheduler}
	\begin{itemize}
		\item The Linux Kernel Scheduler is a key piece in having a real-time behaviour
		\item It is in charge of deciding which \textbf{runnable} task gets executed
		\item It also elects on which CPU the task runs, and is tightly coupled to CPUidle and CPUFreq
		\item It schedules both \textbf{userspace} tasks and \textbf{kernel} tasks
		\item Each task is assigned one \textbf{scheduling class} or \textbf{policy}
		\item The class determines the algorithm used to elect each task
		\item Tasks with different scheduling classes can coexist on the system
	\end{itemize}
\end{frame}

\begin{frame}
	\frametitle{Non-Realtime Scheduling Classes}
	There are 3 \textbf{Non-RealTime} classes
	\begin{itemize}
		\item \code{SCHED_OTHER}: The default policy, using a time-sharing algorithm
		\item \ksym{SCHED_BATCH}: Similar to \code{SCHED_OTHER}, but designed for CPU-intensive loads that affect the wakeup time
		\item \ksym{SCHED_IDLE}: Very low priority class. Tasks with this policy will run only if nothing else needs to run.
		\item \code{SCHED_OTHER} and \ksym{SCHED_BATCH} use the \textbf{nice} value to increase or decrease their scheduling frequency
		\item A higher nice value means that the tasks gets scheduled \textbf{less} often
	\end{itemize}
\end{frame}

\begin{frame}
	\frametitle{Realtime Scheduling Classes}
	There are 3 \textbf{Realtime} classes
	\begin{itemize}
		\item Tasks under a Realtime class can be assigned a priority between 0 and 98
		\item Priority 99 is \textbf{reserved} for critical housekeeping tasks
		\item Runnable tasks will preempt any other lower-priority task
		\item \ksym{SCHED_FIFO}: All tasks with the same priority are scheduled \textbf{First in, First out}
		\item \ksym{SCHED_RR}: Similar to \ksym{SCHED_FIFO} but with a time-sharing round-robin between tasks with the same priority
		\item \ksym{SCHED_DEADLINE}: For tasks doing recurrent jobs, extra attributes are attached to a task
			\begin{itemize}
				\item A computation time, which represents the time the task needs to complete a job
				\item A deadline, which is the maximum allowable time to compute the job
				\item A period, during which only one job can occur
			\end{itemize}
	\end{itemize}
\end{frame}

\begin{frame}
	\frametitle{Changing the Scheduling Class}
	\begin{itemize}
		\item The Scheduling Class is set per-task, and defaults to \code{SCHED_OTHER}
		\item The \manpage{sched_setscheduler}{2} syscall allows changing the class of a task
		\item The tool \code{chrt} uses it to allow changing the class of a running task:
			\begin{itemize}
				\item \code{chrt -f/-b/-o/-r/-d -p PRIO PID}
			\end{itemize}
		\item It can also be used to launch a new program with a dedicated class:
			\begin{itemize}
				\item \code{chrt -f/-b/-o/-r/-d PRIO CMD}
			\end{itemize}
		\item New processes will inherit the class of their parent except if the \ksym{SCHED_RESET_ON_FORK} flag is set with \manpage{sched_setscheduler}{2}
		\item See \manpage{sched}{7} for more information
	\end{itemize}
\end{frame}


\begin{frame}
	\frametitle{Realtime policies and CPU hogging}
	\begin{itemize}
		\item A bug or an infinite loop in a high-priority realtime task will cause the system to become unresponsive
		\item This can be prevented by creating an emergency shell with a higher priority to kill the faulty programs
		\item The Kernel also provides a mechanism to make sure non-RT tasks get to run at one point
		\item \code{/proc/sys/kernel/sched_rt_period_us} defines a window in microseconds that the scheduler will share between RT and non-RT tasks
		\item \code{/proc/sys/kernel/sched_rt_runtime_us} defines how-much of that window is going to be dedicated to RT tasks.
		\item The default values allocates 95\% of the CPU time to RT tasks
		\item This mechanism is also called \textbf{RT Throttling}
	\end{itemize}
\end{frame}

% System ticks and NOHZ
\begin{frame}
	\frametitle{System timer}
	\begin{itemize}
		\item The Scheduler is invoked on a regular basis to perform time-sharing activities
		\item This is sequenced through the \textbf{system ticks}, generated by a high resolution timer
		\item Typically run at 250 Hz (x86) or 100 Hz (arm), but can be configured
		\item Several policies regarding system ticks are available:
		\item \kconfig{CONFIG_HZ_PERIODIC}: Always tick at a given rate. This can introduce small latencies but keep the system responsive.
		\item \kconfig{CONFIG_NO_HZ_IDLE}: Disable the tick when idle, for powersasving
		\item \kconfig{CONFIG_NO_HZ_FULL}: Actively disables ticking even when not idle. Introduces long latencies during context switches.
	\end{itemize}
\end{frame}

% Memory management and allocators

% Writing a driver
\begin{frame}
		\frametitle{Writing a driver}
A few considerations can be taken when writing a driver
		\begin{itemize}
			\item Avoid using \ksym{raw_spinlock_t} unless really necessary
			\item Avoid forcing non-threaded interrupts, unless writing a driver involved in interrupt dispatch
				\begin{itemize}
					\item irqchip, gpio-irq drivers
					\item cpufreq and cpuidle drivers due to scheduler interaction
				\end{itemize}
			\item Beware of DMA bus mastering and other serialized IO buffering
				\begin{itemize}
					\item Certain register writes are buffered until the next register read
				\end{itemize}
		\end{itemize}
\end{frame}
