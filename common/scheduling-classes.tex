\begin{frame}
	\frametitle{The Linux Kernel Scheduler}
	\begin{itemize}
		\item The Linux Kernel Scheduler is a key piece in having a real-time behaviour
		\item It is in charge of deciding which \textbf{runnable} task gets executed
		\item It also elects on which CPU the task runs, and is tightly coupled to CPUidle and CPUFreq
		\item It schedules both \textbf{userspace} tasks and \textbf{kernel} tasks
		\item Each task is assigned one \textbf{scheduling class} or \textbf{policy}
		\item The class determines the algorithm used to elect each task
		\item Tasks with different scheduling classes can coexist on the system
	\end{itemize}
\end{frame}

\begin{frame}
	\frametitle{Non-Realtime Scheduling Classes}
	There are 3 \textbf{Non-RealTime} classes
	\begin{itemize}
		\item \code{SCHED_OTHER}: The default policy, using a time-sharing algorithm
		\item \ksym{SCHED_BATCH}: Similar to \code{SCHED_OTHER}, but designed for CPU-intensive loads that affect the wakeup time
		\item \ksym{SCHED_IDLE}: Very low priority class. Tasks with this policy will run only if nothing else needs to run.
		\item \code{SCHED_OTHER} and \ksym{SCHED_BATCH} use the \textbf{nice} value to increase or decrease their scheduling frequency
		\item A higher nice value means that the tasks gets scheduled \textbf{less} often
	\end{itemize}
\end{frame}

\begin{frame}
	\frametitle{Realtime Scheduling Classes}
	There are 3 \textbf{Realtime} classes
	\begin{itemize}
		\item Tasks under a Realtime class can be assigned a priority between 0 and 98
		\item Priority 99 is \textbf{reserved} for critical housekeeping tasks
		\item Runnable tasks will preempt any other lower-priority task
		\item \ksym{SCHED_FIFO}: All tasks with the same priority are scheduled \textbf{First in, First out}
		\item \ksym{SCHED_RR}: Similar to \ksym{SCHED_FIFO} but with a time-sharing round-robin between tasks with the same priority
		\item \ksym{SCHED_DEADLINE}: For tasks doing recurrent jobs, extra attributes are attached to a task
			\begin{itemize}
				\item A computation time, which represents the time the task needs to complete a job
				\item A deadline, which is the maximum allowable time to compute the job
				\item A period, during which only one job can occur
			\end{itemize}
	\end{itemize}
\end{frame}

\begin{frame}
	\frametitle{Changing the Scheduling Class}
	\begin{itemize}
		\item The Scheduling Class is set per-task, and defaults to \code{SCHED_OTHER}
		\item The \manpage{sched_setscheduler}{2} syscall allows changing the class of a task
		\item The tool \code{chrt} uses it to allow changing the class of a running task:
			\begin{itemize}
				\item \code{chrt -f/-b/-o/-r/-d -p PRIO PID}
			\end{itemize}
		\item It can also be used to launch a new program with a dedicated class:
			\begin{itemize}
				\item \code{chrt -f/-b/-o/-r/-d PRIO CMD}
			\end{itemize}
		\item New processes will inherit the class of their parent except if the \ksym{SCHED_RESET_ON_FORK} flag is set with \manpage{sched_setscheduler}{2}
		\item See \manpage{sched}{7} for more information
	\end{itemize}
\end{frame}
