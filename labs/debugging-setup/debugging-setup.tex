\subchapter
{Preparing the system}
{Objectives:
  \begin{itemize}
    \item Prepare the STM32MP157D board
  \end{itemize}
}

\section{Install needed packages}

You need some development packages before being able to build the target firmware:

\begin{bashinput}
$ sudo apt install build-essential git
\end{bashinput}

\section{Building the image}

We created a special image for the training. This image will contain everything
we need (tools, configured kernel, etc). This image will be built with buildroot
which allows to build a complete image for embedded systems.

\begin{bashinput}
$ cd /home/$USER/debugging-labs/
$ git clone https://github.com/bootlin/buildroot
$ cd buildroot
$ git checkout debugging-training/2022.08
$ make stm32mp157a_dk1_debugging_defconfig
$ make
\end{bashinput}

This will take a few minutes. At the end of the build, a images directory will
contain the images that can be used on the board. During this course, we will
use a kernel located on a sdcard and a rootfs via NFS. This will let us transfer
data freely from and to the target board.

The rootfs should be extracted at \code{/home/$USER/debugging-labs/nfsroot}
using this command:

\begin{bashinput}
$ tar xvf output/images/rootfs.tar -C /home/$USER/debugging-labs/nfsroot
\end{bashinput}

We will also export the \code{CROSS_COMPILE} variable to set the toolchain as our
cross compiling toolchain:

\begin{bashinput}
$ export CROSS_COMPILE=/home/$USER/debugging-labs/buildroot/output/host/bin/arm-linux-
\end{bashinput}

This export needs to be either done in each shell in which \code{CROSS_COMPILE} is
going to be used or added to your shell configuration (\code{.bashrc} for
instance)

\input{../common/stm32-prepare.tex}

\section{Set up the Ethernet communication on the workstation}

With a network cable, connect the Ethernet port of your board to the
one of your computer. If your computer already has a wired connection
to the network, your instructor will provide you with a USB Ethernet
adapter. A new network interface should appear on your Linux system.

Find the name of this interface by typing:
\begin{bashinput}
$ ip a
\end{bashinput}

The network interface name is likely to be
\code{enxxx}\footnote{Following the {\em Predictable Network Interface
Names} convention:
\url{https://www.freedesktop.org/wiki/Software/systemd/PredictableNetworkInterfaceNames/}}.
If you have a pluggable Ethernet device, it's easy to identify as it's
the one that shows up after pluging in the device.

Then, instead of configuring the host IP address from NetworkManager’s graphical interface,
let’s do it through its command line interface, which is so much easier to use:

\begin{bashinput}
$ nmcli con add type ethernet ifname en<xxx> ip4 192.168.0.1/24
\end{bashinput}

\section{Setting up the NFS server}

Install the NFS server by installing the \code{nfs-kernel-server}
package:

\begin{bashinput}
$ sudo apt install nfs-kernel-server
\end{bashinput}

Once installed, edit the \code{/etc/exports} file as
\code{root} to add the following lines, assuming that the IP address
of your board will be \code{192.168.0.100}:

\scriptsize
\begin{bashinput}
/home/<user>/debugging-labs/nfsroot 192.168.0.100(rw,no_root_squash,no_subtree_check)
\end{bashinput}
\normalsize

Of course, replace \code{<user>} by your actual user name.

Make sure that the path and the options are on the same line.
Also make sure that there is no space between the IP address and the NFS
options, otherwise default options will be used for this IP address,
causing your root filesystem to be read-only.

Then, restart the NFS server:

\begin{bashinput}
$ sudo exportfs -r
\end{bashinput}

If there is any error message, this usually means that there was a
syntax error in the \code{/etc/exports} file. Don't proceed until these
errors disappear.

\section{Flashing the sdcard}

In order to get a working board, you will need to flash a sdcard with the 
\code{output/images/sdcard.img} file. Plug your sdcard on your computer and
check on which \code{/dev/sdX} it has been mounted (you can use the \code{dmesg}
command to check that). For instance, if the sdcard has been mounted on
\code{/dev/sde}, use the following command:

\begin{bashinput}
$ sudo dd if=output/images/sdcard.img of=/dev/sde
$ sync
\end{bashinput}

NOTE: Double-check that you are targeting the correct device before executing
the dd command !

Once flashed, plug the sdcard onto the STM32MP157D board and reboot the board.

\section{U-Boot setup}

In order to use a rootfs on NFS, we will use an external rootfs. This can be
specified by passing bootargs to the kernel. To do so, we are going to set the
\code{bootargs} U-Boot variable and save the environment. To be able to edit
U-Boot variables, we need to interrupt its standard boot sequence: with the
serial console opened, maintain Enter key while pressing once the reset button,
and wait for U-Boot prompt to appear. Then, enter the following commands:

\begin{bashinput}
STM32MP1> env set bootargs root=/dev/nfs ip=192.168.0.100:::::eth0
  nfsroot=192.168.0.1:/home/<user>/debugging-labs/nfsroot/,nfsvers=3,tcp rw
STM32MP1> saveenv
\end{bashinput}

NOTE: Be sure to replace the \code{<user>} string with the correct user.

\section{Login}

You can now run the system by pressing the RESET button on the board.

You can login on the serial console or via SSH (\code{ssh
root@192.168.0.100}). The username is \code{root} and the password
is \code{root}.
