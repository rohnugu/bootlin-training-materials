\subchapter{Third party libraries and applications}{Objective: Learn
  how to leverage existing libraries and applications: how to
  configure, compile and install them}

To illustrate how to use existing libraries and applications, we will
extend the small root filesystem built in the {\em A tiny embedded
system} lab to add the {\em ALSA} libraries and tools to run
basic sound support tests, and the {\em libgpiod} library and
executables to manage GPIOs. {\em ALSA} stands for {\em Advanced Linux
Sound Architecture}, and is the Linux audio subsystem.

We'll see that manually re-using existing libraries is quite tedious,
so that more automated procedures are necessary to make it
easier. However, learning how to perform these operations manually
will significantly help you when you face issues with more
automated tools.

\section{Figuring out library dependencies}

We're going to integrate the {\em alsa-utils}, {\em libgpiod}
and {\em ipcalc} executables. In our case, the dependency chain
for {\em alsa-utils} is quite simple, it only depends on the
{\em alsa-lib} library. {\em libgpiod} and {\em ipcalc} are standalone
and don't have any dependency.

\includegraphics[width=\textwidth]{labs/sysdev-thirdparty-stm32/dependencies.pdf}

Of course, all these libraries rely on the C library, which is not
mentioned here, because it is already part of the root filesystem
built in the {\em A tiny embedded system} lab. You might wonder how to
figure out this dependency tree by yourself. Basically, there are
several ways, that can be combined:

\begin{itemize}
\item Read the library documentation, which often mentions the
  dependencies;
\item Read the help message of the \code{configure script} (by running
  \code{./configure --help}).
\item By running the \code{configure} script, compiling and looking
  at the errors.
\end{itemize}

To configure, compile and install all the components of our system,
we're going to start from the bottom of the tree with {\em alsa-lib},
then continue with {\em alsa-utils}. Then, we will also build
{\em libgpiod} and {\em ipcalc}.

\section{Preparation}

For our cross-compilation work, we will need two separate spaces:
\begin{itemize}
\item A \emph{staging} space in which we will directly install all the
  packages: non-stripped versions of the libraries, headers,
  documentation and other files needed for the compilation. This
  \emph{staging} space can be quite big, but will not be used on our
  target, only for compiling libraries or applications;
\item A \emph{target} space, in which we will only copy the required
  files from the \emph{staging} space: binaries and libraries, after
  stripping, configuration files needed at runtime, etc. This target
  space will take a lot less space than the \emph{staging} space, and
  it will contain only the files that are really needed to make the
  system work on the target.
\end{itemize}

To sum up, the {\em staging} space will contain everything that's
needed for compilation, while the {\em target} space will contain only
what's needed for execution.

Create the \code{$HOME/__SESSION_NAME__-labs/thirdparty} directory,
and inside, create two directories: \code{staging} and \code{target}.

For the target, we need a basic system with BusyBox and
initialization scripts. We will re-use the system built in the {\em A
  tiny embedded system} lab, so copy this system in the target
directory:

\bashcmd{$ cp -a $HOME/__SESSION_NAME__-labs/tinysystem/nfsroot/* target/}

Note that for this lab, a lot of typing will be required. To save time
typing, we advise you to copy and paste commands from the electronic
version of these instructions.

\section{Testing}

Make sure the \code{target/} directory is exported by your NFS server
to your board by modifying \code{/etc/exports} and restarting your NFS
server.

Make your board boot from this new directory through NFS.

\section{alsa-lib}

{\em alsa-lib} is a library supposed to handle the interaction with
the ALSA subsystem. It is available at \url{https://alsa-project.org}.
Download version 1.2.9, and extract it
in \code{$HOME/__SESSION_NAME__-labs/thirdparty/}.

{\bf Tip}: if the website for any of the source packages that we
need to download in the next sections is down, a great mirror
that you can use is \url{http://sources.buildroot.net/}.

Back to {\em alsa-lib} sources, look at the \code{configure} script
and see that it has been generated by \code{autoconf} (the header
contains a sentence like {\em Generated by GNU Autoconf 2.69}). Most of
the time, \code{autoconf} comes with \code{automake}, that generates
Makefiles from \code{Makefile.am} files. So {\em alsa-lib} uses a rather
common build system. Let's try to configure and build it:

\begin{bashinput}
$ ./configure
$ make
\end{bashinput}

If you look at the generated binaries, you'll see that they are
x86 ones because we compiled the sources with gcc, the default compiler.
This is obviously not what we want, so let's clean-up the generated objects
and tell the \code{configure} script to use the ARM cross-compiler:

\begin{bashinput}
$ make clean
$ CC=arm-linux-gcc ./configure
\end{bashinput}

Of course, the \code{arm-linux-gcc} cross-compiler must be in your
\code{PATH} prior to running the configure script. The \code{CC} environment
variable is the classical name for specifying the compiler to
use.

Quickly, you should get an error saying:

%\footnotesize
\begin{terminaloutput}
checking whether we are cross compiling... configure: error: in `/home/tux/__SESSION_NAME__-labs/thirdparty/alsa-lib-1.2.9':
configure: error: cannot run C compiled programs.
If you meant to cross compile, use `--host'.
See `config.log' for more details
\end{terminaloutput}
\normalsize

If you look at the \code{config.log} file, you can see that the
\code{configure} script compiles a binary with the cross-compiler
and then tries to run it on the development workstation. This is a
rather usual thing to do for a \code{configure} script, and that's
why it tests so early that it's actually doable, and bails out if not.

Obviously, it cannot work in our case, and the scripts exits. The job
of the \code{configure} script is to test the configuration of the system. To
do so, it tries to compile and run a few sample applications to test
if this library is available, if this compiler option is supported,
etc. But in our case, running the test examples is definitely not
possible.

We need to tell the \code{configure} script that we are cross-compiling, and
this can be done using the \code{--build} and \code{--host} options,
as described in the help of the \code{configure} script:

\begin{verbatim}
System types:
  --build=BUILD	configure for building on BUILD [guessed]
  --host=HOST	cross-compile to build programs to run on HOST [BUILD]
\end{verbatim}

The \code{--build} option allows to specify on which system the
package is built, while the \code{--host} option allows to specify on
which system the package will run. By default, the value of the
\code{--build} option is guessed and the value of \code{--host} is the
same as the value of the \code{--build} option. The value is guessed
using the \code{./config.guess} script, which on your system should
return \code{x86_64-pc-linux-gnu}. See
\url{https://www.gnu.org/software/autoconf/manual/html_node/Specifying-Names.html}
for more details on these options.

So, let's override the value of the \code{--host} option:

\bashcmd{$ ./configure --host=arm-linux}

Note that \code{CC} is not required anymore. It is implied
by \code{--host}.

The \code{configure} script should end properly now, and create a
Makefile.

However, there is one subtle issue to handle.
We need to tell {\em alsa-lib} to disable a feature called {\em alsa topology}.
{\em alsa-lib} will build fine but we will encounter some
problems afterwards, during {\em alsa-utils} building.
So you should configure {\em alsa-lib} as follows:

\bashcmd{$ ./configure --host=arm-linux --disable-topology}

Run the \code{make} command, which should run just fine.

Look at the result of compiling in \code{src/.libs}: a set of object files
and a set of \code{libasound.so*} files.

The \code{libasound.so*} files are a dynamic version of the
library. The shared library itself is \code{libasound.so.2.0.0}, it has
been generated by the following command line:

\begin{bashinput}
$ arm-linux-gcc -shared conf.o confmisc.o input.o output.o async.o error.o dlmisc.o socket.o shmarea.o userfile.o names.o -lm -ldl -lpthread -lrt -Wl,-soname -Wl,libasound.so.2 -o libasound.so.2.0.0
\end{bashinput}

And creates the symbolic links \code{libasound.so} and
\code{libasound.so.2}.

\begin{bashinput}
$ ln -s libasound.so.2.0.0 libasound.so.2
$ ln -s libasound.so.2.0.0 libasound.so
\end{bashinput}

These symlinks are needed for two different reasons:

\begin{itemize}
\item \code{libasound.so} is used at compile time when you want to
  compile an application that is dynamically linked against the
  library. To do so, you pass the \code{-lLIBNAME} option to the
  compiler, which will look for a file named
  \code{lib<LIBNAME>.so}. In our case, the compilation option is
  \code{-lasound} and the name of the library file is
  \code{libasound.so}. So, the \code{libasound.so} symlink is needed
  at compile time;
\item \code{libasound.so.2} is needed because it is the {\em SONAME}
  of the library. {\em SONAME} stands for {\em Shared Object Name}. It
  is the name of the library as it will be stored in applications
  linked against this library. It means that at runtime, the dynamic
  loader will look for exactly this name when looking for the shared
  library. So this symbolic link is needed at runtime.
\end{itemize}

To know what's the {\em SONAME} of a library, you can use:
\bashcmd{$ arm-linux-readelf -d libasound.so.2.0.0}

and look at the \code{(SONAME)} line. You'll also see that this
library needs the C library, because of the \code{(NEEDED)} line on
\code{libc.so.0}.

The mechanism of \code{SONAME} allows to change the library without
recompiling the applications linked with this library. Let's say that
a security problem is found in the {\em alsa-lib} release that provides
{\em libasound 2.0.0}, and fixed in the next {\em alsa-lib} release, which will
now provide {\em libasound 2.0.1}.

You can just recompile the library, install it on your target system,
change the \code{libasound.so.2} link so that it points to
\code{libasound.so.2.0.1} and restart your applications. And it will
work, because your applications don't look specifically for
\code{libasound.so.2.0.0} but for the {\em SONAME}
\code{libasound.so.2}.

However, it also means that as a library developer, if you break the
ABI of the library, you must change the {\em SONAME}: change from
\code{libasound.so.2} to \code{libasound.so.3}.

Finally, the last step is to tell the \code{configure} script where the
library is going to be installed. Most \code{configure} scripts consider that
the installation prefix is \code{/usr/local/} (so that the library is
installed in \code{/usr/local/lib}, the headers in
\code{/usr/local/include}, etc.). But in our system, we simply want
the libraries to be installed in the \code{/usr} prefix, so let's tell
the \code{configure} script about this:

\begin{bashinput}
$ ./configure --host=arm-linux --disable-topology --prefix=/usr
$ make
\end{bashinput}

For this library, this option may not change anything to the resulting
binaries, but for safety, it is always recommended to make sure that
the prefix matches where your library will be running on the target
system.

Do not confuse the {\em prefix} (where the application or library will
be running on the target system) from the location where the
application or library will be installed on your host while building
the root filesystem.

For example, {\em libasound} will be installed in
\code{$HOME/__SESSION_NAME__-labs/thirdparty/target/usr/lib/} because this is
the directory where we are building the root filesystem, but once our
target system will be running, it will see {\em libasound} in
\code{/usr/lib}.

The prefix corresponds to the path in the target system and {\bf
  never} on the host. So, one should {\bf never} pass a prefix like
\code{$HOME/__SESSION_NAME__-labs/thirdparty/target/usr}, otherwise at
runtime, the application or library may look for files inside this
directory on the target system, which obviously doesn't exist! By
default, most build systems will install the application or library in
the given prefix (\code{/usr} or \code{/usr/local}), but with most
build systems (including {\em autotools}), the installation prefix can
be overridden, and be different from the configuration prefix.

We now only have the installation process left to do.

First, let's make the installation in the {\em staging} space:
\bashcmd{$ make DESTDIR=$HOME/__SESSION_NAME__-labs/thirdparty/staging install}

Now look at what has been installed by {\em alsa-lib}:
\begin{itemize}
\item Some configuration files in \code{/usr/share/alsa}
\item The headers in \code{/usr/include}
\item The shared library and its libtool (\code{.la}) file in \code{/usr/lib}
\item A pkgconfig file in \code{/usr/lib/pkgconfig}. We'll come back
  to these later
\end{itemize}

Finally, let's install the library in the {\em target} space:

\begin{enumerate}
\item Create the \code{target/usr/lib} directory, it will contain the
  stripped version of the library
\item Copy the dynamic version of the library. Only
  \code{libasound.so.2} and \code{libasound.so.2.0.0} are needed,
  since \code{libasound.so.2} is the {\em SONAME} of the library and
  \code{libasound.so.2.0.0} is the real binary:
  \begin{itemize}
  \item \bashcmd{$ cp -a staging/usr/lib/libasound.so.2* target/usr/lib}
  \end{itemize}
\item Measure the size of the \code{target/usr/lib/libasound.so.2.0.0}
  library before stripping.
\item Strip the library:
  \begin{itemize}
  \item \bashcmd{$ arm-linux-strip target/usr/lib/libasound.so.2.0.0}
  \end{itemize}
\item Measure the size of the \code{target/usr/lib/libasound.so.2.0.0}
  library library again after stripping. How many unnecessary bytes
  were saved?
\end{enumerate}

Then, we need to install the {\em alsa-lib} configuration files:

\begin{bashinput}
$ mkdir -p target/usr/share
$ cp -a staging/usr/share/alsa target/usr/share
\end{bashinput}

Now, we need to adjust one small detail in one of the configuration
files. Indeed, \code{/usr/share/alsa/alsa.conf} assumes a UNIX group
called \code{audio} exists, which is not the case on our very small
system. So edit this file, and replace \code{defaults.pcm.ipc_gid
audio} by \code{defaults.pcm.ipc_gid 0} instead.

And we're done with {\em alsa-lib}!

\section{Alsa-utils}

Download {\em alsa-utils} from the ALSA offical webpage. We tested the lab
with version 1.2.9.

Once uncompressed, we quickly discover that the {\em alsa-utils} build
system is based on the {\em autotools}, so we will work once again
with a regular \code{configure} script.

As we've seen previously, we will have to provide the prefix and host
options and the CC variable:

\bashcmd{$ ./configure --host=arm-linux --prefix=/usr}

Now, we should quiclky get an error in the execution of the
\code{configure} script:

\begin{verbatim}
checking for libasound headers version >= 1.2.5 (1.2.5)... not present.
configure: error: Sufficiently new version of libasound not found.
\end{verbatim}

Again, we can check in \code{config.log} what the \code{configure}
script is trying to do:

%\footnotesize
\begin{terminaloutput}
configure:15855: checking for libasound headers version >= 1.2.5 (1.2.5)
configure:15902: arm-linux-gcc -c -g -O2  conftest.c >&5
conftest.c:24:10: fatal error: alsa/asoundlib.h: No such file or directory
\end{terminaloutput}
\normalsize

Of course, since {\em alsa-utils} uses {\em alsa-lib}, it includes
its header file! So we need to tell the C compiler where the headers
can be found: there are not in the default directory
\code{/usr/include/}, but in the \code{/usr/include} directory of our
{\em staging} space. The help text of the \code{configure} script says:

\begin{verbatim}
  CPPFLAGS    (Objective) C/C++ preprocessor flags, e.g. -I<include dir> if
              you have headers in a nonstandard directory <include dir>
\end{verbatim}

Let's use it:

\begin{bashinput}
$ CPPFLAGS=-I$HOME/__SESSION_NAME__-labs/thirdparty/staging/usr/include \%\linebreak
./configure --host=arm-linux --prefix=/usr
\end{bashinput}

Now, it should stop a bit later, this time with the error:
\begin{verbatim}
checking for snd_ctl_open in -lasound... no
configure: error: No linkable libasound was found.
\end{verbatim}

The \code{configure} script tries to compile an application against {\em
  libasound} (as can be seen from the \code{-lasound} option): {\em
  alsa-utils} uses {\em alsa-lib}, so the \code{configure} script
wants to make sure this library is already installed. Unfortunately,
the \code{ld} linker doesn't find it. So, let's tell the
linker where to look for libraries using the \code{-L} option followed
by the directory where our libraries are (in
\code{staging/usr/lib}). This \code{-L} option can be passed to the
linker by using the \code{LDFLAGS} at configure time, as told by the
help text of the \code{configure} script:

\begin{verbatim}
  LDFLAGS     linker flags, e.g. -L<lib dir> if you have libraries in a
              nonstandard directory <lib dir>
\end{verbatim}

Let's use this \code{LDFLAGS} variable:

\begin{bashinput}
$ LDFLAGS=-L$HOME/__SESSION_NAME__-labs/thirdparty/staging/usr/lib \
     CPPFLAGS=-I$HOME/__SESSION_NAME__-labs/thirdparty/staging/usr/include \
     ./configure --host=arm-linux --prefix=/usr
\end{bashinput}

Once again, it should fail a bit further down the tests, this time
complaining about a missing {\em curses helper header}. {\em curses}
or {\em ncurses} is a graphical framework to design UIs in the
terminal. This is only used by {\em alsamixer}, one of the tools
provided by {\em alsa-utils}, that we are not going to use.
Hence, we can just disable the build of {\em alsamixer}.

Of course, if we wanted it, we would have had to build {\em ncurses} first,
just like we built {\em alsa-lib}.

\begin{bashinput}
$ LDFLAGS=-L$HOME/__SESSION_NAME__-labs/thirdparty/staging/usr/lib \
     CPPFLAGS=-I$HOME/__SESSION_NAME__-labs/thirdparty/staging/usr/include \
     ./configure --host=arm-linux --prefix=/usr \
     --disable-alsamixer
\end{bashinput}

Then, run the compilation with \code{make}. You should hit a final
error:

\begin{verbatim}
Making all in po
make[2]: Entering directory '/home/tux/embedded-linux-labs/thirdparty/alsa-utils-1.2.9/alsaconf/po'
mv: cannot stat 't-ja.gmo': No such file or directory
\end{verbatim}

This can be fixed by disabling support for \code{alsaconf} too:

\begin{bashinput}
$ LDFLAGS=-L$HOME/__SESSION_NAME__-labs/thirdparty/staging/usr/lib \
     CPPFLAGS=-I$HOME/__SESSION_NAME__-labs/thirdparty/staging/usr/include \
     ./configure --host=arm-linux --prefix=/usr \
     --disable-alsamixer --disable-alsaconf
\end{bashinput}

You can now run \code{make} again.  It should work this time.

Let's now begin the installation process.  Before really installing in
the staging directory, let's install in a dummy directory, to see
what's going to be installed (this dummy directory will not be used
afterwards, it is only to verify what will be installed before
polluting the staging space):

\bashcmd{$ make DESTDIR=/tmp/alsa-utils/ install}

The \code{DESTDIR} variable can be used with all Makefiles based on
\code{automake}. It allows to override the installation directory:
instead of being installed in the configuration prefix directory, the
files will be installed in \code{DESTDIR/configuration-prefix}.

Now, let's see what has been installed in \code{/tmp/alsa-utils/} (run
\code{tree /tmp/alsa-utils}):

\begin{verbatim}
/tmp/alsa-utils/
├── lib
│   ├── systemd
│   │   └── system
│   │       ├── alsa-restore.service
│   │       ├── alsa-state.service
│   │       └── sound.target.wants
│   │           ├── alsa-restore.service -> ../alsa-restore.service
│   │           └── alsa-state.service -> ../alsa-state.service
│   └── udev
│       └── rules.d
│           └── 90-alsa-restore.rules
├── usr
│   ├── bin
│   │   ├── aconnect
│   │   ├── alsabat
│   │   ├── alsaloop
│   │   ├── alsatplg
│   │   ├── alsaucm
│   │   ├── amidi
│   │   ├── amixer
│   │   ├── aplay
│   │   ├── aplaymidi
│   │   ├── arecord -> aplay
│   │   ├── arecordmidi
│   │   ├── aseqdump
│   │   ├── aseqnet
│   │   ├── axfer
│   │   ├── iecset
│   │   └── speaker-test
│   ├── lib
│   │   └── alsa-topology
│   │       ├── libalsatplg_module_nhlt.la
│   │       └── libalsatplg_module_nhlt.so
│   ├── sbin
│   │   ├── alsabat-test.sh
│   │   ├── alsaconf
│   │   ├── alsactl
│   │   └── alsa-info.sh
│   └── share
│       ├── alsa
│       │   └── init
│       │       ├── 00main
│       │       ├── ca0106
│       │       ├── default
│       │       ├── hda
│       │       ├── help
│       │       ├── info
│       │       └── test
│       ├── man
│       │   ├── fr
│       │   │   └── man8
│       │   │       └── alsaconf.8
│       │   ├── man1
│       │   │   ├── aconnect.1
│       │   │   ├── alsabat.1
│       │   │   ├── alsactl.1
│       │   │   ├── alsa-info.sh.1
│       │   │   ├── alsaloop.1
│       │   │   ├── amidi.1
│       │   │   ├── amixer.1
│       │   │   ├── aplay.1
│       │   │   ├── aplaymidi.1
│       │   │   ├── arecord.1 -> aplay.1
│       │   │   ├── arecordmidi.1
│       │   │   ├── aseqdump.1
│       │   │   ├── aseqnet.1
│       │   │   ├── axfer.1
│       │   │   ├── axfer-list.1
│       │   │   ├── axfer-transfer.1
│       │   │   ├── iecset.1
│       │   │   └── speaker-test.1
│       │   ├── man7
│       │   └── man8
│       │       └── alsaconf.8
│       └── sounds
│           └── alsa
│               ├── Front_Center.wav
│               ├── Front_Left.wav
│               ├── Front_Right.wav
│               ├── Noise.wav
│               ├── Rear_Center.wav
│               ├── Rear_Left.wav
│               ├── Rear_Right.wav
│               ├── Side_Left.wav
│               └── Side_Right.wav
└── var
    └── lib
        └── alsa

25 directories, 63 files
\end{verbatim}

So, we have:
\begin{itemize}
\item The {\em systemd} service definitions in \code{lib/systemd}
\item The {\em udev} rules in \code{lib/udev}
\item The {\em alsa-utils} binaries in \code{/usr/bin} and \code{/usr/sbin}
\item Some sound samples in \code{/usr/share/sounds}
\item The various translations in \code{/usr/share/locale}
\item The manual pages in \code{/usr/share/man/}, explaining how to
  use the various tools
\item Some configuration samples in \code{/usr/share/alsa}.
\end{itemize}

Now, let's make the installation in the {\em staging} space:

\bashcmd{$ make DESTDIR=$HOME/__SESSION_NAME__-labs/thirdparty/staging/ install}

Then, let's manually install only the necessary files in the {\em target}
space. We are only interested in \code{speaker-test}:

\begin{bashinput}
$ cd ..
$ cp -a staging/usr/bin/speaker-test target/usr/bin/
$ arm-linux-strip target/usr/bin/speaker-test
\end{bashinput}

And we're finally done with {\em alsa-utils}!

Now test that all is working fine by running the \code{speaker-test} util on
your board, with the headset provided by your instructor plugged
in. You may need to add the missing libraries from the toolchain
install directory.

Now you can use:

\begin{itemize}

\item \code{speaker-test} with no arguments to generate {\em pink noise}

\item \code{speaker-test -t sine} to generate a {\em sine wave},
optionally with \code{-f <freq>} for a specific frequency

\end{itemize}

There you are: you built and ran your first program depending
on a library different from the C library.

\section{libgpiod}

\subsection{Compiling libgpiod}

We are now going to use {\em libgpiod} (instead of the
deprecated interface in \code{/sys/class/gpio}, whose executables
(\code{gpiodetect}, \code{gpioset}, \code{gpioget}...) will
allow us to drive and manage GPIOs from shell scripts.

Here, we will be using the 2.0.x version of {\em libgpiod}.

\begin{bashinput}
git clone https://git.kernel.org/pub/scm/libs/libgpiod/libgpiod.git
cd libgpiod
git checkout v2.0.x
\end{bashinput}

As we are not starting from a release, we will need to install
further development tools to generate some files like the
\code{configure} script:

\bashcmd{sudo apt install autoconf-archive pkg-config}

Now let's generate the files which are present in a release:

\bashcmd{./autogen.sh}

Run \code{./configure --help} script, and see that this script provides
a \code{--enable-tools} option which allows to build the userspace
executables that we want.

As this project doesn't have any external library dependency, let's
configure {\em libgpiod} in a similar way as {\em alsa-utils}:

\begin{bashinput}
$ ./configure --host=arm-linux --prefix=/usr --enable-tools
\end{bashinput}

Now, compile the software:

\bashcmd{$ make}

Installation to the {\em staging} space can be done using the
classical \code{DESTDIR} mechanism:

\begin{bashinput}
$ make DESTDIR=$HOME/__SESSION_NAME__-labs/thirdparty/staging/ install
\end{bashinput}

And finally, only manually install and strip the files
needed at runtime in the {\em target} space:

\begin{bashinput}
$ cd ..
$ cp -a staging/usr/lib/libgpiod.so.3* target/usr/lib/
$ arm-linux-strip target/usr/lib/libgpiod*
$ cp -a staging/usr/bin/gpio* target/usr/bin/
$ arm-linux-strip target/usr/bin/gpio*
\end{bashinput}

\subsection{Testing libgpiod}

First, connect \code{GPIO PE1} (pin D2 of connector CN14) connected
to ground (pin 7 of connector CN16), as in the
{\em Accessing Hardware Devices} lab.

Now, let's run the \code{gpiodetect} command on the target, and check that
you can list the various GPIO banks on your system.

\begin{bashinput}
# gpiodetect
gpiochip0 [GPIOA] (16 lines)
gpiochip1 [GPIOB] (16 lines)
gpiochip2 [GPIOC] (16 lines)
gpiochip3 [GPIOD] (16 lines)
gpiochip4 [GPIOE] (16 lines)
gpiochip5 [GPIOF] (16 lines)
gpiochip6 [GPIOG] (16 lines)
gpiochip7 [GPIOH] (16 lines)
gpiochip8 [GPIOI] (12 lines)
gpiochip9 [GPIOZ] (8 lines)
\end{bashinput}

Again, we can see that GPIOE is represented by \code{gpiochip4}.

We can then get details on GPIOE GPIOs by running \code{gpioinfo
gpiochip4} or on all GPIOs by simply running \code{gpioinfo}.

You can now read the state of your GPIO PE1:

\begin{bashinput}
# gpioget -c gpiochip4 1
"1"=inactive
\end{bashinput}

Now, connect your wire to 3V3 (pin 2 of connector CN16). You should now
read:

\begin{bashinput}
# gpioget -c gpiochip4 1
"1"=active
\end{bashinput}

You see that you didn't have to configure the GPIO as input. {\em
libgpiod} did that for you.

If you have an LED and a small breadboard (or M-F breadboard wires),
you could also try to drive the GPIO in output mode. Connect the short
pin of the LED to GND, and the long one to the GPIO. Then then following
command should light up the diode:

\begin{bashinput}
# gpioset -c gpiochip4 1=1
\end{bashinput}

Here's how to turn it off:

\begin{bashinput}
# gpioset -c gpiochip4 1=0
\end{bashinput}

\code{gpioset} offers many more options. Run \code{gpioset -h} to
check by yourself.

\section{ipcalc}

After practicing with autotools based packages, let's build {\em ipcalc}, which
is using {\em Meson} as build system. We won't really need this utility
in our system, but at least it has no dependencies and therefore
offers an easy way to build our first {\em Meson} based package.

So, first install the \code{meson} package:

\bashcmd{$ sudo apt install meson}

In the main lab directory, then let's check out the sources through
\code{git}:

\begin{bashinput}
$ git clone https://gitlab.com/ipcalc/ipcalc.git
$ cd ipcalc/
$ git checkout 1.0.3
\end{bashinput}

To cross-compile with {\em Meson}, we need to create a {\em cross file}.
Let's create the \code{../cross-file.txt} file with the below contents:

\begin{verbatim}
[binaries]
c = 'arm-linux-gcc'

[host_machine]
system = 'linux'
cpu_family = 'arm'
cpu = 'cortex-a7'
endian = 'little'
\end{verbatim}

We also need to create a special directory for building:

\begin{bashinput}
$ mkdir cross-build
$ cd cross-build
\end{bashinput}

We can now have \code{meson} create the Ninja build files for us:

\begin{bashinput}
$ meson --cross-file ../../cross-file.txt --prefix /usr ..
\end{bashinput}

We are now ready to build {\em ipcalc}:

\begin{bashinput}
$ ninja
\end{bashinput}

And now install \code{ipcalc} to the build space:

\begin{bashinput}
$ DESTDIR=$HOME/__SESSION_NAME__-labs/thirdparty/staging ninja install
\end{bashinput}

Check that the \code{staging/usr/bin/ipcalc} file is indeed an ARM
executable.

The last thing to do is to copy it to the target space and strip it:

\begin{bashinput}
$ cd ../..
$ cp staging/usr/bin/ipcalc target/usr/bin/
$ arm-linux-strip target/usr/bin/ipcalc
\end{bashinput}

Note that we could have asked \code{ninja install} to strip the
executable for us when installing it into the staging directory.
To do, this, we would have added a \code{strip} entry in the cross file,
and passed \code{--strip} to {\em Meson}. However, it's better to keep
files unstripped in the staging space, in case we need to debug them.

You can now test that \code{ipcalc} works on the target:

\begin{bashinput}
# ipcalc 192.168.0.100
Address:	192.168.0.100
Address space:	Private Use
\end{bashinput}

\section{Final touch}

To finish this lab completely, and to be consistent with what we've done before,
let's strip the C library and its loader too.

First, check the initial size of the binaries:
\bashcmd{$ ls -l target/lib}

Then strip the binaries in \code{/lib}:
\begin{bashinput}
$ chmod +w target/lib/*.so.*
$ arm-linux-strip target/lib/*.so.*
\end{bashinput}

And check the final size:
\bashcmd{$ ls -l target/lib/}
