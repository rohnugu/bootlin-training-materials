\subchapter{Lab9: Using devtool}{Automate recipe development and debugging
  using devtool}

During this lab, you will:
\begin{itemize}
  \item Use devtool to generate a new recipe more quickly
  \item Modify a recipe to add a new patch using devtool
  \item Upgrade a recipe to a newer version using devtool
\end{itemize}

\section{Generate a new recipe}

The \code{devtool} executable is already available in your shell after
sourcing the \code{oe-init-build-env} script. Take a look at the
sub-commands it offers:
\begin{bashinput}
devtool --help
\end{bashinput}

We now want to add a new recipe for the ``GNU Hello'' program
(\url{https://www.gnu.org/software/hello/}). We can do so using the
\code{add} subcommand.

However we want to use version 2.10 instead of the latest mainline version,
so we can use the \code{--version} option:
\begin{bashinput}
devtool add --version 2.10 https://ftp.gnu.org/gnu/hello/hello-2.10.tar.gz
\end{bashinput}

You can observe that \code{devtool} calls \code{bitbake} multiple times. In the output messages, some lines are particularly interesting:

\begin{bashinput}
INFO: Creating workspace layer in .../build/workspace
...
INFO: Using default source tree path .../build/workspace/sources/hello
...
INFO: Recipe .../build/workspace/recipes/hello/hello_2.10.bb has been automatically created; further editing may be required to make it fully functional
\end{bashinput}

The first \code{INFO} line means devtool has created the workspace in the
\code{workspace} directory inside the \code{build} directory. Take a moment
to inspect how the workspace looks like, but remind to never modify it
manually: the workspace is the internal state of devtool, so it may stop
working if it is modified externally. The workspace now looks like this:

\begin{bashinput}
$ tree workspace/ | head -n20
workspace/
|-- README
|-- appends
|   `-- hello_2.10.bbappend
|-- conf
|   `-- layer.conf
|-- recipes
|   `-- hello
|       `-- hello_2.10.bb
`-- sources
    `-- hello
        ...
        |-- GNUmakefile
        |-- INSTALL
        |-- Makefile.am
        |-- Makefile.in
$
\end{bashinput}

As you can see the workspace is a layer, having a \code{conf/layer.conf}
file. It also has a directory for recipes which already holds a recipe for
GNU Hello 2.10. The \code{sources} directory contains the source code of
the hello recipe, that devtool uses internally to manage patches.

You can see that the workspace layer has been enabled by checking your
\code{conf/bblayers.conf} or by running \code{bitbake-layers show-layers}.

It's time to try building the GNU Hello program via devtool:

\begin{bashinput}
devtool build hello
\end{bashinput}

In the output messages, look for:
\begin{bashinput}
NOTE: hello: compiling from external source tree .../workspace/sources/hello
\end{bashinput}

This means that the recipes managed by devtool do not download the source
code in the usual way, but rather they use the local copy of the sources
that has been previously populated is in
\code{workspace/sources/<RECIPENAME>}.

You can have a look at the output of the build, which is in the usual
location inside the workdir:
\code{tmp/work/<ARCHITECTURE>/hello/2.10-r0/image/}.

Using \code{devtool deploy-target} is a handy way to try the newly built
code on the target:

\begin{bashinput}
devtool deploy-target hello root@192.168.0.100
\end{bashinput}

This will send to the device all the file in the \code{image} subdirectory
of the recipe work directory, keeping the directory layout and file
permissions. You can now test the program on the target:

\begin{bashinput}
$ ssh root@192.168.0.100
root@bootlinlabs:~# hello
Hello, world!
root@bootlinlabs:~#
\end{bashinput}

Before moving the recipe to the meta-bootlinlabs layer, have a look at the
recipe code as it has been generated by devtool:

\begin{bashinput}
devtool edit-recipe hello
\end{bashinput}

This command will open the \code{hello_2.10.bb} file of the workspace in an
editor. You can see that the recipe has various comments added by devtool:
you should review them, fix or adapt whatever is needed, and save the
recipe.

First of all, check the exact license by reading the first few lines of
\code{workspace/sources/hello/src/hello.c} and you will discover it is a
``GPL 3.0 or later''. The license guessed by devtool is
\code{GPL-3.0-only}, thus replace it by \code{GPL-3.0-or-later}.

Devtool has already computed the hashes for you, but there are several
\code{SRC_URI} hashed, thus feel free to remove all of them except
\code{SRC_URI[sha256sum]}.

You can also simplify the \code{SRC_URI} line using \code{GNU_MIRROR},
getting:
\begin{verbatim}
  SRC_URI = "${GNU_MIRROR}/hello/hello-${PV}.tar.gz"
\end{verbatim}

Note the \code{inherit} line: from the content of the source code files of
the GNU Hello program, devtool already guessed that it is configured using
the Autotools and it is using GNU Gettext. This saved us a lot of time!
There is a comment above the \code{inherit} line: do not remove it for the
moment.

Finally there is a line setting \code{EXTRA_OECONF} to an empty
string. This line is useless unless you know you need to set some
configuration flags, thus you can remove it together with the comment line.

The resulting recipe (\code{workspace/recipes/hello/hello_2.10.bb}) should
now look like:
\begin{verbatim}
  LICENSE = "GPL-3.0-or-later"
  LIC_FILES_CHKSUM = "file://COPYING;md5=d32239bcb673463ab874e80d47fae504"

  SRC_URI = "${GNU_MIRROR}/hello/hello-${PV}.tar.gz"
  SRC_URI[sha256sum] = "31e066137a962676e89f69d1b65382de95a7ef7d914b8cb956f41ea72e0f516b"

  # NOTE: if this software is not capable of being built in a separate build directory
  # from the source, you should replace autotools with autotools-brokensep in the
  # inherit line
  inherit gettext autotools
\end{verbatim}

You can double check that your recipe still works as expected using
\code{devtool deploy-target} and when you are done you can remove the files
from the target:
\begin{bashinput}
devtool undeploy-target hello root@192.168.0.100
\end{bashinput}

You can now stop having the hello recipe handled by devtool and move it to
the meta-bootlinlabs layer:
\begin{bashinput}
devtool finish -f hello ../meta-bootlinlabs/
\end{bashinput}

Now the recipe is in
\code{../meta-bootlinlabs/recipes-hello/hello/hello_2.10.bb}. You can read
the .bb file and verify the content has not changed. Move the recipe to a
more reasonable directory name:
\begin{bashinput}
mv ../meta-bootlinlabs/recipes-hello ../meta-bootlinlabs/recipes-utils
\end{bashinput}

Now check the content of the workspace: the hello recipe is not there
anynmore. However the source code of the GNU Hello program as still in the
\code{workspace/sources/hello} directory. Devtool does not delete it, in
case you have done any valuable work in it that you still haven't saved to
a patch. As it is not your case, just delete it:
\begin{bashinput}
rm -fr workspace/sources/hello/
\end{bashinput}

Now double check that the recipe is still building correctly in the
meta-bootlinlabs layer. There's no reason it should fail, right?
\begin{bashinput}
bitbake hello
\end{bashinput}

Oops, it failed! To have a hint about the reason, have a look at the
comment we didn't remove from the recipe:
\begin{verbatim}
# NOTE: if this software is not capable of being built in a separate build directory
# from the source, you should replace autotools with autotools-brokensep in the
# inherit line
\end{verbatim}

This points exactly to the problem with GNU Hello 2.10: it fails building
out-of-tree, i.e. with a build directory different from the source
directory, as is done by default when using \code{autotools.bbclass}. Just
fix the recipe as the comment suggests by changing the \code{inherit}
line. You can then remove the comment as well. Your work dir is now
polluted so you need to clean it before running a new build:
\begin{bashinput}
bitbake -c clean hello
bitbake hello
\end{bashinput}

That's all: you now have a very concise (and working!) hello recipe in the
meta-bootlinlabs layer!

\section{Modify a recipe}

Now use devtool to modify the hello recipe adding a patch. This can be very
useful if you have to fix a bug in a third-party program and there is now
patch around to fix it yet.

Use \code{devtool modify} to put an existing recipe under the control of devtool:
\begin{bashinput}
devtool modify hello
\end{bashinput}

Now you have two hello recipes: one in the meta-bootlinlabs layers and one
in the workspace layer. To ensure about which will be used by bitbake, you
can inspect the layer priorities.

Now enter the source directory. You can notice devtool created a git
repository into it:

\begin{bashinput}
cd workspace/sources/hello/
git log
\end{bashinput}

The generated git repository contains only one commit which contains
thepristine sources as extracted from the downloaded tarball.  Now open the
\code{src/hello.c} with an editor, go around line 60 and edit the ``Hello,
world'' string to print whatever you prefer. Save and exit the
editor. Check your modification using \code{git diff -- src/hello.c} and
test it as done earlier:

\begin{bashinput}
devtool build hello
devtool deploy-target hello root@192.168.0.100
\end{bashinput}

Edit again the source code if needed. When you are satisfied with your
changes, just commit them using git:

\begin{bashinput}
git add src/hello.c
git commit -m 'Change the greeting message'
\end{bashinput}

Check your changes in the git repository, then exit the workspace:

\begin{bashinput}
git log
cd $BUILDDIR
\end{bashinput}

You are now ready to update the original recipe to take into account your
changes:

\begin{bashinput}
devtool update-recipe hello
\end{bashinput}

Looking at the \code{recipes-utils/hello} directory in the meta-bootlinlabs
layer you can notice that a new patch has been created and added to
\code{SRC_URI}. The patch applies the same change that you have just
committed. With devtool you don't need to handle the patch syntax, but
rather you can use git as you are probably already used to.

Now remove the hello recipe from under the control of devtool, then cleanup
as done earlier:

\begin{bashinput}
devtool reset hello
rm -fr workspace/sources/hello/
\end{bashinput}

\section{Upgrade a recipe to the latest mainline version}

devtool can be used to simplify the upgrade of a recipe to a newer mainline
version.

First, it can detect which is the latest version available on the original
site. This is based on heuristics that work for projects that store their
source tarballs in a standard way, which most do. To check for the latest
version, use:
\begin{bashinput}
devtool latest-version hello
\end{bashinput}

At the time of this writing, the output shows that 2.12.1 is the latest
version. You can modify the recipe to use the newest version, including the
computation of the new hashes and renaming the .bb file, with a single
command:
\begin{bashinput}
devtool upgrade hello
\end{bashinput}

Note the INFO line about license changes that you have to verify. Double
check in the sources that the license is still ``GPL 3.0 or later'', then
open the recipe as it is currently in the devtool workspace:
\begin{bashinput}
devtool edit-recipe hello
\end{bashinput}

The recipe file contains a diff between the old and the new version of the
recipe. If you can see that the license changes are not relevant, as is the
case for the upgrade from 2.10 to 2.12.1, then you can simply delete the
comment, save and exit.

Now check that everything works as done earlier, then apply your changes to the layer.
\begin{bashinput}
devtool finish hello ../meta-bootlinlabs/
rm -fr workspace/sources/hello/
\end{bashinput}

Now the meta-bootlinlabs layer contains the latest version of GNU Hello!
