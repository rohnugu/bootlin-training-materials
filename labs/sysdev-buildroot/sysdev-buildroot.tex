\subchapter{Using a build system, example with Buildroot}{Objectives:
  discover how a build system is used and how it works, with the
  example of the Buildroot build system. Build a full Linux system,
  including the Linux kernel.}

\section{Goals}

Compared to the previous lab, we are going to build a more elaborate
system, still containing {\em alsa-utils} (and of course its {\em
alsa-lib} dependency), but this time using Buildroot,
an automated build system.

The automated build system will also allow us to add more packages
and play real audio on our system, thanks to the {\em Music Player
Daemon (mpd)} (\url{https://www.musicpd.org/} and its {\em mpc} client.

As in a real project, we will also build the Linux kernel from
Buildroot, and install the kernel modules in the root filesystem.

\section{Setup}

Go to the \code{$HOME/__SESSION_NAME__-labs/buildroot} directory.

\section{Get Buildroot and explore the source code}

The official Buildroot website is available at
\url{https://buildroot.org/}. Clone the {\em Git} repository:

\begin{bashinput}
git clone https://git.buildroot.net/buildroot
cd buildroot
\end{bashinput}

Now checkout the tag corresponding to the latest 2023.02.<n> release (Long
Term Support), which we have tested for this lab.

Several subdirectories or files are visible, the most important ones
are:

\begin{itemize}
\item \code{boot} contains the Makefiles and configuration items
  related to the compilation of common bootloaders (GRUB, U-Boot,
  Barebox, etc.)
\item \code{board} contains board specific configurations and
  root filesystem overlays.
\item \code{configs} contains a set of predefined configurations,
  similar to the concept of defconfig in the kernel.
\item \code{docs} contains the documentation for Buildroot.
\item \code{fs} contains the code used to generate the various root
  filesystem image formats
\item \code{linux} contains the Makefile and configuration items
  related to the compilation of the Linux kernel
\item \code{Makefile} is the main Makefile that we will use to use
  Buildroot: everything works through Makefiles in Buildroot;
\item \code{package} is a directory that contains all the Makefiles,
  patches and configuration items to compile the user space
  applications and libraries of your embedded Linux system. Have a
  look at various subdirectories and see what they contain;
\item \code{system} contains the root filesystem skeleton and the {\em
    device tables} used when a static \code{/dev} is used;
\item \code{toolchain} contains the Makefiles, patches and
  configuration items to generate the cross-compiling toolchain.
\end{itemize}

\section{Board specific configuration}

As we will want Buildroot to build a kernel with a custom configuration,
and our custom patch, so let's add our own subdirectory under
\code{board}:

\begin{bashinput}
mkdir -p board/bootlin/training
\end{bashinput}

Then, copy your kernel configuration and kernel patch:

\begin{bashinput}
cp ../../kernel/linux/.config board/bootlin/training/linux.config
cp ../../kernel/linux/0001-Custom-DTS-for-Bootlin-lab.patch \
   board/bootlin/training/
\end{bashinput}

We will configure Buildroot to use this kernel configuration.

\section{Configure Buildroot}

In our case, we would like to:

\begin{itemize}
\item Generate an embedded Linux system for ARM;
\item Use an already existing external toolchain instead of having
  Buildroot generating one for us;
\item Compile the Linux kernel and deploy its modules in the root
  filesystem;
\item Integrate {\em BusyBox}, {\em alsa-utils},
  {\em mpd}, {\em mpc} and {\em evtest} in our embedded Linux system;
\item Integrate the target filesystem into a tarball
\end{itemize}

To run the configuration utility of Buildroot, simply run:

\bashcmd{$ make menuconfig}

Set the following options. Don't hesitate to press the \code{Help}
button whenever you need more details about a given option:

\begin{itemize}
\item \code{Target options}
  \begin{itemize}
  \ifdefstring{\labboard}{beagleplay}{
  \item \code{Target Architecture}: \code{AArch64 (little endian)}
  }{
  \item \code{Target Architecture}: \code{ARM (little endian)}
  }
   \ifdefstring{\labboard}{stm32mp1}{
   \item \code{Target Architecture Variant}: \code{cortex-A7}
  }{
   \ifdefstring{\labboard}{qemu}{
   \item \code{Target Architecture Variant}: \code{cortex-A9}
   \item \code{Enable NEON SIMD extension support}: Enabled
   \item \code{Enable VFP extension support}: Enabled
   }{%beaglebone or beagleplay
    \ifdefstring{\labboard}{beagleplay}{
    \item \code{Target Architecture Variant}: \code{cortex-A53}
    }{
    \item \code{Target Architecture Variant}: \code{cortex-A8}
    }
   }
  }
  \ifdefstring{\labboard}{beagleplay}{}{
  \item \code{Target ABI}: \code{EABIhf}
  }
  \ifdefstring{\labboard}{stm32mp1}{
  \item \code{Floating point strategy}: \code{VFPv4}
  }{
   \ifdefstring{\labboard}{qemu}{
   \item \code{Floating point strategy}: \code{VFPv3-D16}
   }{%beaglebone or beagleplay
    \ifdefstring{\labboard}{beagleplay}{
    \item \code{Floating point strategy}: \code{FP-ARMv8}
    }{
    \item \code{Floating point strategy}: \code{VFPv3-D16}
    }
   }
  }
  \end{itemize}
\item \code{Toolchain}
  \begin{itemize}
  \item \code{Toolchain type}: \code{External toolchain}
  \item \code{Toolchain}: \code{Custom toolchain}
  \item \code{Toolchain path}: use the toolchain you built:
  \ifdefstring{\labboard}{beagleplay}{
    \code{/home/<user>/x-tools/aarch64-training-linux-musl}
    }{
      \code{/home/<user>/x-tools/arm-training-linux-musleabihf}
    }
    (replace \code{<user>} by your actual user name)
  \item \code{External toolchain gcc version}: \code{12.x}
  \item \code{External toolchain kernel headers series}: \ifdefstring{\labboard}{beagleplay}{\texttt{6.1 or later}}{\texttt{\workingkernel.x or later}}
  \item \code{External toolchain C library}: \code{musl (experimental)}
  \item We must tell Buildroot about our toolchain configuration, so
    select \code{Toolchain has SSP support?} and
    \code{Toolchain has C++ support?}.
    Buildroot will check these parameters anyway.
  \end{itemize}
\item \code{Kernel}
  \begin{itemize}
  \item Enable \code{Linux Kernel}
  \if\defstring{\labboard}{beagleplay}
  \item Set \code{Kernel version} to \code{Custom version}
  \item Set \code{Kernel version} to \code{6.6.20}
  \else
  \item Set \code{Kernel version} to \code{Latest version (6.1)}
  \fi
  \item Set \code{Custom kernel patches} to \code{board/bootlin/training/0001-Custom-DTS-for-Bootlin-lab.patch}
  \item Set \code{Kernel configuration} to \code{Using a custom (def)config file})
  \item Set \code{Configuration file path} to \code{board/bootlin/training/linux.config}
  \item Select \code{Build a Device Tree Blob (DTB)}
  \item Set \code{In-tree Device Tree Source file names} to
        \ifdefstring{\labboard}{stm32mp1}{\code{stm32mp157a-dk1-custom}}{}
        \ifdefstring{\labboard}{beaglebone}{\code{am335x-boneblack-custom}}{}
        \ifdefstring{\labboard}{beagleplay}{\code{ti/k3-am625-beagleplay-custom}}{}
  \ifdefstring{\labboard}{beagleplay}{\item Set \code{Kernel Binary format} to \code{Image.gz}}{}
\end{itemize}
\item \code{Target packages}
  \begin{itemize}
  \item Keep \code{BusyBox} (default version) and keep the BusyBox
    configuration proposed by Buildroot;
  \item \code{Audio and video applications}
    \begin{itemize}
    \item Select \code{alsa-utils}, and in the submenu:
    \begin{itemize}
         \item Only keep \code{speaker-test}
    \end{itemize}
    \item Select \code{mpd}, and in the submenu:
    \begin{itemize}
         \item Keep only \code{alsa}, \code{vorbis} and \code{tcp sockets}
    \end{itemize}
    \item Select \code{mpd-mpc}.
    \end{itemize}
  \item \code{Hardware handling}
    \begin{itemize}
	 \item Select \code{evtest}\\
	       This userspace application allows to test events from
	       input devices. This way, we will be able to test the
	       Nunchuk by getting details about which buttons were
	       pressed.
    \end{itemize}
  \end{itemize}
\item \code{Filesystem images}
  \begin{itemize}
  \item Select \code{tar the root filesystem}
  \end{itemize}
\end{itemize}

Exit the menuconfig interface. Your configuration has now been saved
to the \code{.config} file.

\section{Generate the embedded Linux system}

Just run:

\bashcmd{$ make}

Buildroot will first create a small environment with the external
toolchain, then download, extract, configure, compile and install each
component of the embedded system.

All the compilation has taken place in the \code{output/} subdirectory. Let's
explore its contents:

\begin{itemize}

\item \code{build}, is the directory in which each component built by
  Buildroot is extracted, and where the build actually takes place

\item \code{host}, is the directory where Buildroot installs some
  components for the host. As Buildroot doesn't want to depend on too
  many things installed in the developer machines, it installs some
  tools needed to compile the packages for the target. In our case it
  installed {\em pkg-config} (since the version of the host may be ancient)
  and tools to generate the root filesystem image ({\em genext2fs},
  {\em makedevs}, {\em fakeroot}).

\item \code{images}, which contains the final images produced by
  Buildroot. In our case it contains a tarball of the filesystem, called
  \code{rootfs.tar}, plus the compressed kernel and Device Tree binary.
  Depending on the configuration, there could also a bootloader binary
  or a full SD card image.

\item \code{staging}, which contains the “build” space of the target
  system. All the target libraries, with headers and documentation. It
  also contains the system headers and the C library, which in our
  case have been copied from the cross-compiling toolchain.

\item \code{target}, is the target root filesystem. All applications
  and libraries, usually stripped, are installed in this
  directory. However, it cannot be used directly as the root
  filesystem, as all the device files are missing: it is not possible
  to create them without being root, and Buildroot has a policy of not
  running anything as root.

\end{itemize}

\section{Run the generated system}

Go back to the \code{$HOME/__SESSION_NAME__-labs/buildroot/} directory. Create
a new \code{nfsroot} directory that is going to hold our system,
exported over NFS. Go into this directory, and untar the rootfs using:

\bashcmd{$ tar xvf ../buildroot/output/images/rootfs.tar}

Add our \code{nfsroot} directory to the list of directories exported
by NFS in \code{/etc/exports}.

Also update the kernel and Device Tree binaries used by your board,
from the ones compiled by Buildroot in \code{output/images/}.

Boot the board, and log in (\code{root} account, no password).

You should now reach a shell.

\section{Loading the USB audio module}

You can check that no kernel module is loaded yet. Try to load the
\code{snd_usb_audio} module from the command line.

This should work. Check that Buildroot has deployed the modules
for your kernel in \code{/lib/modules}.

Let's automate this now!

Look at the \code{/etc/inittab} file generated by Buildroot (ask your
instructor if you have any questions), and at the contents of the
\code{/etc/init.d/} directory, in particular of the \code{rcS} file.

You can see that \code{rcS} executes or sources all the \code{/etc/init.d/S??*}
files. We can add our own which will load the toplevel modules that we
need.

Let's do this by creating an {\em overlay directory}, typically under
our board specific directory, that Buildroot will add after building the
root filesystem:

\begin{bashinput}
mkdir -p board/bootlin/training/rootfs-overlay/
\end{bashinput}

Then add a custom startup script, by adding an \code{etc/init.d/S03modprobe}
executable file to the overlay directory, with the below contents:

\begin{verbatim}
#!/bin/sh
modprobe snd-usb-audio
\end{verbatim}

Then, go back to Buildroot's configuration interface:
\begin{itemize}
\item \code{System configuration}
  \begin{itemize}
  \item Set \code{Root filesystem overlay directories} to
        \code{board/bootlin/training/rootfs-overlay}
  \end{itemize}
\end{itemize}

Build your image again. This should be quick as Buildroot doesn't need
to recompile anything. It will just apply the root filesystem overlay.

Update your \code{nfsroot} directory, reboot the board and check
that the \code{snd_usb_audio} module is loaded as expected.

You can run \code{speaker-test} to check that audio indeed works.

\section{Testing music playback with mpd and mpc}

The next thing we want to do is play real sound samples with
the {\em Music Player Daemon (MPD)}. So, let's add music files
\footnote{For the most part, these are public domain
music files, except a small sample file... See the \code{README.txt}
file in the directory containing the files.}
for MPD to play:

\begin{bashinput}
mkdir -p board/bootlin/training/rootfs-overlay/var/lib/mpd/music
cp ../data/music/* board/bootlin/training/rootfs-overlay/var/lib/mpd/music
\end{bashinput}

Update your root filesystem. Thanks to NFS, you don't need to restart
your system.

Using the \code{ps} command, check that the \code{mpd} server
was started by the system, as implemented by the
\code{/etc/init.d/S95mpd} script.

If that's the case, you are now ready to run \code{mpc} client commands
to control music playback. First, let's make \code{mpd} process the
newly added music files. Run this command on the target:

\bashcmd{# mpc update}

You should see the files getting indexed, by displaying the contents
of the \code{/var/log/mpd.log} file:

\begin{terminaloutput}
Jan 01 00:04 : exception: Failed to open '/var/lib/mpd/state': No such file or directory
Jan 01 00:15 : update: added /2-arpent.ogg
Jan 01 00:15 : update: added /6-le-baguette.ogg
Jan 01 00:15 : update: added /4-land-of-pirates.ogg
Jan 01 00:15 : update: added /3-chronos.ogg
Jan 01 00:15 : update: added /1-sample.ogg
Jan 01 00:15 : update: added /7-fireworks.ogg
Jan 01 00:15 : update: added /5-ukulele-song.ogg
\end{terminaloutput}

You can also check the list of available files:
\begin{terminaloutput}
# mpc listall
1-sample.ogg
2-arpent.ogg
5-ukulele-song.ogg
3-chronos.ogg
7-fireworks.ogg
6-le-baguette.ogg
4-land-of-pirates.ogg
\end{terminaloutput}

To play files, you first need to create a playlist. Let's create a
playlist by adding all music files to it:

\bashcmd{# mpc add /}

You should now be able to start playing the songs in the playlist:

\bashcmd{# mpc play}

Here are a few further commands for controlling playback:
\begin{itemize}
\item \code{mpc volume +5}: increase the volume by 5\%
\item \code{mpc volume -5}: reduce the volume by 5\%
\item \code{mpc prev}: switch to the previous song in the playlist.
\item \code{mpc next}: switch to the next song in the playlist.
\item \code{mpc toggle}: toggle between pause and playback modes.
\end{itemize}

If you find that changing the volume is not available, you can
add a custom configuration for MPD, as the standard one
provided by Buildroot doesn't support allowing to change the audio
playback volume with all sound cards we have tested. We will simply
add this file to our overlay:

\begin{bashinput}
cp ../data/mpd.conf board/bootlin/training/rootfs-overlay/etc/
\end{bashinput}

Run Buildroot again and update your root filesystem. Here again, you
don't need to reboot. It's sufficient to restart MPD to make it read
the new configuration file:

\bashcmd{# /etc/init.d/S95mpd restart}

You can now make sure that modifying the volume works.

Later, we will compile and debug a custom MPD client application.

\section{Analyzing dependencies}

It's always useful to understand the dependencies drawn by the
packages we build.

First we need to install a {\em Graphviz}:

\bashcmd{$ sudo apt install graphviz}

Now, let's use Buildroot's target to generate a
dependency graph:

\bashcmd{$ make graph-depends}

We can now study the dependency graph:

\bashcmd{$ evince output/graphs/graph-depends.pdf}

In particular, you can see that adding MPD and its client
required to compile {\em Meson} for the host, and in turn,
{\em Python 3} for the host too. This substantially contributed to the
build time.

\section{Adding a Buildroot package}

We would also like to build our Nunchuk external module with Buildroot.
Fortunately, Buildroot has a \code{kernel-module} infrastructure
to build kernel modules.

First, create a \code{nunchuk-driver} subdirectory under \code{package}
in Buildroot sources.

The first thing is to create a \code{package/nunchuk-driver/Config.in} file
for Buildroot's configuration:

\begin{verbatim}
config BR2_PACKAGE_NUNCHUK_DRIVER
        bool "nunchuk-driver"
        depends on BR2_LINUX_KERNEL
        help
                Linux Kernel module for the I2C Nunchuk.
\end{verbatim}

Then add a line to \code{package/Config.in} to include this file, for example right
before the line including \code{package/nvidia-driver/Config.in},
so that the alphabetic order of configuration options is kept.

Then, the next and last thing you need to do is create
\code{package/nunchuk-driver/nunchuk-driver.mk} describing how to build
the package:

\begin{verbatim}
NUNCHUK_DRIVER_VERSION = 1.0
NUNCHUK_DRIVER_SITE = $(HOME)/__SESSION_NAME__-labs/hardware/data/nunchuk
NUNCHUK_DRIVER_SITE_METHOD = local
NUNCHUK_DRIVER_LICENSE = GPL-2.0

$(eval $(kernel-module))
$(eval $(generic-package))
\end{verbatim}

Then, configure Buildroot to build your package, run Buildroot and
update your root filesystem.

Can you load the \code{nunchuk} module now? If everything's fine, add a
line to \code{/etc/init.d/S03modprobe} for this driver, and update your
root filesystem once again.

\section{Testing the Nunchuk}

Now that we have the \code{nunchuk} driver loaded and that Buildroot compiled
\code{evtest} for the target, thanks to Buildroot, we can now test the input
events coming from the Nunchuk.


\if\defstring{\labboard}{beagleplay}
\begin{bashinput}
# evtest
No device specified, trying to scan all of /dev/input/event*
Available devices:
/dev/input/event0:	gpio-keys
/dev/input/event1:	GN Netcom A/S Jabra EVOLVE 20 MS
/dev/input/event2:	GN Netcom A/S Jabra EVOLVE 20 MS
/dev/input/event3:	GN Netcom A/S Jabra EVOLVE 20 MS
/dev/input/event4:	GN Netcom A/S Jabra EVOLVE 20 MS Consumer Control
/dev/input/event5:	Wii Nunchuk
Select the device event number [0-5]:
\end{bashinput}
\else
\begin{bashinput}
# evtest
No device specified, trying to scan all of /dev/input/event*
Available devices:
/dev/input/event0:	pmic_onkey
/dev/input/event1:	Logitech Inc. Logitech USB Headset H340 Consumer Control
/dev/input/event2:	Logitech Inc. Logitech USB Headset H340
/dev/input/event3:	Wii Nunchuk
Select the device event number [0-3]:
\end{bashinput}
\fi

Enter the number corresponding to the Nunchuk device.

You can now press the Nunchuk buttons, use the joypad, and see which
input events are emitted.

By the way, you can also test which input events are exposed by the
driver for your audio headset (if any), which doesn't mean that they physically
exist.

\section{Commit your changes}

As we are going to reuse our Buildroot changes in the next labs,
let's commit them into a branch:

\begin{bashinput}
git checkout -b bootlin-labs
git add board/bootlin/ package/nunchuk-driver/
git commit -as -m "Bootlin lab changes"
\end{bashinput}

\section{Going further}

{\em If you finish your lab before the others}

\begin{itemize}
\item For more music playing fun, you can install the \code{ario} or
  {\em cantata} MPD client on your host machine (\code{sudo apt
  install ario}, \code{sudo apt install cantata}), configure it to
  connect to the IP address of your target system with the default
  port, and you will also be able to control playback from your host
  machine.
\end{itemize}

