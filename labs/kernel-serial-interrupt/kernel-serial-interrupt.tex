\subchapter{Sleeping and handling interrupts}{Objective: learn how to
  register and implement a simple interrupt handler, and how to put a
  process to sleep and wake it up at a later point}

During this lab, you will:

\begin{itemize}
\item Register an interrupt handler for the serial controller of the
  Beaglebone.
\item Implement the read() operation of the serial port driver to put
  the process to sleep when no data are available.
\item Implement the interrupt handler to wake-up the sleeping process
  waiting for received characters.
\item Handle communication between the interrupt handler and the
  read() operation.
\end{itemize}

\section{Setup}

This lab is a continuation of the {\em Output-only misc driver
lab}. Use the same kernel, environment and paths!

\section{Register the handler}

Declare an interrupt handler function stub. Then, in the module probe
function, we need to register this handler, binding it to the right
IRQ number.

Nowadays, Linux is using a virtual IRQ number that it derives from the
hardware interrupt number. This virtual number is created through the
\code{irqdomain} mechanism. The hardware IRQ number to use is found in
the device tree.

First, retrieve the unique IRQ number to request:

\begin{verbatim}
int irq;
irq = platform_get_irq(pdev, 0);
\end{verbatim}

Then, pass the interrupt number to \kfunc{devm_request_irq} along
with the interrupt handler to register your interrupt in the kernel.

Then, in the interrupt handler, just print a message and return
\ksym{IRQ_HANDLED} (to tell the kernel that we have handled the
interrupt).

You'll also need to enable interrupts.
To do so, in the \code{probe()} function, write
\ksym{UART_IER_RDI} to the \ksym{UART_IER} register.

Compile and load your module. Send a character on the serial link (just
type something in the corresponding \code{picocom} terminal, and
look at the kernel logs: they are full of our message indicating that
interrupts are occurring, even if we only sent one character! It shows
you that interrupt handlers should do a little bit more when an
interrupt occurs.

\section{Enable and filter the interrupts}

In fact, the hardware will replay the interrupt until you acknowledge
it. Linux will only dispatch the interrupt event to the rightful
handler, hoping that this handler will acknowledge it. What we
experienced here is called an \code{interrupt flood}.

Now, in our interrupt handler, we want to acknowledge the
interrupt. On the UART controllers that we drive, it's done simply by
reading the contents of the \ksym{UART_RX} register, which holds the
next character received. You can display the value you read to see
that the driver will receive whatever character you sent.

Compile and load your driver. Have a look at the kernel messages. You
should no longer be flooded with interrupt messages. In the kernel
log, you should see the message of our interrupt handler. If not,
check your code once again and ask your instructor for clarification!

Load and unload your driver multiple times, to make sure that
there are no registration / deregistration issues.

\section{Sleeping, waking up and communication}

Now, we would like to implement the \code{read()} operation of our
driver so that a user space application reading from our device can
receive the characters from the serial port.

First, we need a communication mechanism between the interrupt handler
and the \code{read()} operation. We will implement a very simple
circular buffer. So let's add a device-specific buffer to our
\code{serial_dev} structure.

Let's also add two integers that will contain the next location
in the circular buffer that we can write to, and the next location
we can read from:

\begin{verbatim}
#define SERIAL_BUFSIZE 16

struct serial_dev {
        void __iomem *regs;
        struct miscdevice miscdev;
        unsigned int counter
        char rx_buf[SERIAL_BUFSIZE];
        unsigned int buf_rd;
        unsigned int buf_wr;
};
\end{verbatim}

In the interrupt handler, store the received character at location
\code{buf_wr} in the circular buffer, and increment the value
of \code{buf_wr}. If this value reaches \code{SERIAL_BUFSIZE},
reset it to zero.

In the \code{read()} operation, if the \code{buf_rd} value is
different from the \code{buf_wr} value, it means that one
character can be read from the circular buffer. So, read this
character, store it in the user space buffer, update the
\code{buf_rd} variable, and return to user space (we will only
read one character at a time, even if the user space application
requested more than one).

Now, what happens in our \code{read()} function if no character is
available for reading (i.e., if \code{buf_wr} is equal to
\code{buf_rd})? We should put the process to sleep!

To do so, add a \ksym{wait_queue_head_t} wait queue to our
\code{serial_dev} structure, named for example \code{wait}. In the
\code{read()} function, keep things simple by directly using
\kfunc{wait_event_interruptible} right from the start, to wait until
\code{buf_wr} is different from \code{buf_rd}\footnote{A single test in the
\kfunc{wait_event_interruptible} function is sufficient. If the
condition is met, you don't go to sleep and read one character right away.
Otherwise, when you wake up, you can proceed to the reading part.}.

Last but not least, in the interrupt handler,
after storing the received characters in the circular buffer, use
\kfunc{wake_up} to wake up all processes waiting on the wait queue.

Compile and load your driver. Run \code{cat /dev/serial-<...>}
on the target, and then in \code{picocom} on the development
workstation side, type some characters. They should appear on the
remote side if everything works correctly!

Don't be surprised if the keys you type in Picocom don't appear on the
screen. This happens because they are not echoed back by the target.
